% Created 2025-11-16 Sun 18:07
% Intended LaTeX compiler: pdflatex
\documentclass[11pt,a4paper]{article}
\usepackage[utf8]{inputenc}
\usepackage[T1]{fontenc}
\usepackage{graphicx}
\usepackage{longtable}
\usepackage{wrapfig}
\usepackage{rotating}
\usepackage[normalem]{ulem}
\usepackage{amsmath}
\usepackage{amssymb}
\usepackage{capt-of}
\usepackage{hyperref}
\usepackage[utf8]{inputenc}
\usepackage[brazil, american]{babel}
\usepackage{amsmath,amssymb}
\usepackage{tikz}
\usepackage{graphicx}
\usepackage{xcolor}
\usepackage{geometry}
\geometry{margin=2cm}
\usepackage{fancyhdr}
\pagestyle{fancy}
\fancyhead[L]{SERIADO UFMG}
\fancyhead[R]{Período de Férias - Aprofundamento (26 Nov - 02 Dez)}
\author{Material de Estudo SERIADO UFMG}
\date{2025-11-16}
\title{Período de Férias - Aprofundamento (26 Nov - 02 Dez)}
\hypersetup{
 pdfauthor={Material de Estudo SERIADO UFMG},
 pdftitle={Período de Férias - Aprofundamento (26 Nov - 02 Dez)},
 pdfkeywords={},
 pdfsubject={},
 pdfcreator={Emacs 29.4 (Org mode 9.6.15)}, 
 pdflang={English}}
\begin{document}

\maketitle
\setcounter{tocdepth}{2}
\tableofcontents


\section{Material de Estudo - Período de Férias (26/11 - 02/12)}
\label{sec:org5f6d454}
\textbf{Objetivo:} Revisão leve e aprofundamento em tópicos-chave (2h/dia)

\noindent\rule{\textwidth}{0.5pt}

\section{11/26 - Férias Dia 1}
\label{sec:org2c809ae}
\subsection{Aula 30 - Matemática: Função Quadrática - Parte 2 (Zeros, Máximo/Mínimo, Estudo do Sinal) - 90min}
\label{sec:orgf0fe7c8}
\subsubsection{Revisão: Função Quadrática Parte 1}
\label{sec:orge8ced9a}
Na Aula 26, estudamos: - Definição: f(x) = ax² + bx + c - Gráfico: parábola - Concavidade (a > 0: ∪; a < 0: ∩) - Discriminante (Δ) e número de raízes - Fórmula de Bhaskara - Vértice básico

\textbf{Nesta aula:} Aprofundar zeros, máximo/mínimo e estudo do sinal.

\subsubsection{Zeros da Função Quadrática (Revisão Aprofundada)}
\label{sec:orgbca0a52}
\textbf{Zeros (raízes):} valores de x onde f(x) = 0.

\paragraph{Métodos para Encontrar Zeros}
\label{sec:org591c9a0}
\textbf{1. Fórmula de Bhaskara (método geral):}

\begin{verbatim}
Δ = b² - 4ac

x = (-b ± √Δ) / 2a
\end{verbatim}

\textbf{2. Soma e Produto (Relações de Girard):}

\begin{verbatim}
x₁ + x₂ = -b/a
x₁ · x₂ = c/a
\end{verbatim}

\textbf{Uso:} Se conhecemos a soma e produto, podemos encontrar as raízes sem Bhaskara.

\textbf{Exemplo:} Raízes somam 7 e multiplicam 10.

x₁ + x₂ = 7 x₁ · x₂ = 10

Equação do 2º grau: x² - (soma)x + (produto) = 0 x² - 7x + 10 = 0

Raízes: 2 e 5 (verificar: 2+5=7, 2×5=10 ✓)

\textbf{3. Fatoração (quando possível):}

ax² + bx + c = a(x - x₁)(x - x₂)

\textbf{Exemplo:} x² - 5x + 6 = 0 (x - 2)(x - 3) = 0 x = 2 ou x = 3

\textbf{4. Completamento de quadrados:}

Útil para encontrar vértice e zeros simultaneamente.

\textbf{Exemplo:} x² - 4x + 3 = 0 x² - 4x = -3 x² - 4x + 4 = -3 + 4 (x - 2)² = 1 x - 2 = ±1 x = 3 ou x = 1

\subsubsection{Máximo e Mínimo da Função}
\label{sec:org6e3839b}
O vértice da parábola é ponto de \textbf{máximo} ou \textbf{mínimo} da função.

\paragraph{Vértice: V(xᵥ, yᵥ)}
\label{sec:org0ee5e0f}
\textbf{Coordenadas:}

\begin{verbatim}
xᵥ = -b / 2a

yᵥ = -Δ / 4a
ou
yᵥ = f(xᵥ)
\end{verbatim}

\textbf{Interpretação:}

\textbf{Se a > 0 (parábola para cima ∪):} - Vértice é ponto de \textbf{MÍNIMO} - yᵥ = valor mínimo da função - f(x) ≥ yᵥ para todo x

\textbf{Se a < 0 (parábola para baixo ∩):} - Vértice é ponto de \textbf{MÁXIMO} - yᵥ = valor máximo da função - f(x) ≤ yᵥ para todo x

\paragraph{Imagem da Função}
\label{sec:org3bce116}
\textbf{Se a > 0:} Im(f) = [yᵥ, +∞) = \{y ∈ ℝ | y ≥ yᵥ\}

\textbf{Se a < 0:} Im(f) = (-∞, yᵥ] = \{y ∈ ℝ | y ≤ yᵥ\}

\subsubsection{Estudo do Sinal da Função Quadrática}
\label{sec:org3d4c99e}
\textbf{Estudar o sinal:} determinar para quais valores de x a função é positiva, negativa ou zero.

\textbf{Método:} 1. Calcular as raízes (se existirem) 2. Observar a concavidade (sinal de a) 3. Analisar os intervalos

\paragraph{Caso 1: Δ > 0 (duas raízes distintas: x₁ e x₂)}
\label{sec:org54936ce}
Assumindo x₁ < x₂:

\textbf{Se a > 0 (parábola ∪):}

\begin{verbatim}
f(x) > 0: x < x₁ ou x > x₂  (fora das raízes)
f(x) = 0: x = x₁ ou x = x₂  (nas raízes)
f(x) < 0: x₁ < x < x₂        (entre as raízes)
\end{verbatim}

\textbf{Gráfico:}

\begin{verbatim}
 +    +
 \___/
x₁   x₂

+ + + 0 - - - 0 + + +
    x₁     x₂
\end{verbatim}

\textbf{Se a < 0 (parábola ∩):}

\begin{verbatim}
f(x) > 0: x₁ < x < x₂        (entre as raízes)
f(x) = 0: x = x₁ ou x = x₂  (nas raízes)
f(x) < 0: x < x₁ ou x > x₂  (fora das raízes)
\end{verbatim}

\textbf{Gráfico:}

\begin{verbatim}
x₁/‾‾‾\x₂
 -    -

- - - 0 + + + 0 - - -
    x₁     x₂
\end{verbatim}

\paragraph{Caso 2: Δ = 0 (uma raiz: x₁ = x₂ = xᵥ)}
\label{sec:org17ce36a}
\textbf{Se a > 0:}

\begin{verbatim}
f(x) > 0: x ≠ xᵥ  (para todo x exceto xᵥ)
f(x) = 0: x = xᵥ  (só no vértice)
f(x) < 0: nunca
\end{verbatim}

\textbf{Se a < 0:}

\begin{verbatim}
f(x) > 0: nunca
f(x) = 0: x = xᵥ  (só no vértice)
f(x) < 0: x ≠ xᵥ  (para todo x exceto xᵥ)
\end{verbatim}

\paragraph{Caso 3: Δ < 0 (sem raízes reais)}
\label{sec:org3ce151b}
\textbf{Se a > 0:}

\begin{verbatim}
f(x) > 0: para todo x ∈ ℝ
f(x) = 0: nunca
f(x) < 0: nunca
\end{verbatim}

\textbf{Se a < 0:}

\begin{verbatim}
f(x) > 0: nunca
f(x) = 0: nunca
f(x) < 0: para todo x ∈ ℝ
\end{verbatim}

\subsubsection{Inequações do 2º Grau}
\label{sec:org343d894}
Resolver inequações usando estudo do sinal.

\textbf{Tipos:} - ax² + bx + c > 0 - ax² + bx + c ≥ 0 - ax² + bx + c < 0 - ax² + bx + c ≤ 0

\textbf{Método:} 1. Igualar a zero e encontrar raízes 2. Esboçar parábola (concavidade) 3. Identificar região pedida

\textbf{Exemplo 1:} Resolver: x² - 5x + 6 < 0

\textbf{Passo 1: Raízes} x² - 5x + 6 = 0 (x - 2)(x - 3) = 0 x₁ = 2, x₂ = 3

\textbf{Passo 2: Concavidade} a = 1 > 0 → ∪

\textbf{Passo 3: Esboço}

\begin{verbatim}
\___/
2   3
\end{verbatim}

\textbf{Passo 4: f(x) < 0 (região negativa)} Entre as raízes: 2 < x < 3

\textbf{Resposta:} S = \{x ∈ ℝ | 2 < x < 3\} ou (2, 3)

\textbf{Exemplo 2:} Resolver: -x² + 4x - 3 ≥ 0

\textbf{Passo 1: Raízes} -x² + 4x - 3 = 0 Multiplicando por -1: x² - 4x + 3 = 0 (x - 1)(x - 3) = 0 x₁ = 1, x₂ = 3

\textbf{Passo 2: Concavidade} a = -1 < 0 → ∩

\textbf{Passo 3: Esboço}

\begin{verbatim}
1/‾‾‾\3
 -   -
\end{verbatim}

\textbf{Passo 4: f(x) ≥ 0 (região positiva ou zero)} Entre as raízes (incluindo): 1 ≤ x ≤ 3

\textbf{Resposta:} S = [1, 3]

\subsubsection{Exercícios Resolvidos}
\label{sec:org48c737c}
\paragraph{Exercício 1}
\label{sec:orge8cc93f}
Determine o valor máximo de f(x) = -2x² + 8x - 6.

\textbf{Solução:} a = -2 < 0 → tem máximo

xᵥ = -8/2(-2) = -8/(-4) = 2

yᵥ = f(2) = -2(4) + 8(2) - 6 = -8 + 16 - 6 = 2

\textbf{Resposta:} Valor máximo = 2 (em x = 2)

\paragraph{Exercício 2}
\label{sec:org0608006}
Para quais valores de x a função f(x) = x² - 6x + 8 é negativa?

\textbf{Solução:}

\textbf{Raízes:} Δ = 36 - 32 = 4 x = (6 ± 2)/2 x₁ = 2, x₂ = 4

\textbf{Concavidade:} a = 1 > 0 → ∪

\textbf{f(x) < 0:} entre as raízes

\textbf{Resposta:} 2 < x < 4 ou x ∈ (2, 4)

\paragraph{Exercício 3}
\label{sec:org25707e0}
Resolva: x² - 4x + 4 ≥ 0

\textbf{Solução:}

\textbf{Raízes:} Δ = 16 - 16 = 0 x = 4/2 = 2 (raiz dupla)

\textbf{Concavidade:} a = 1 > 0 → ∪

\textbf{Análise:} Δ = 0 e a > 0 - Toca o eixo x apenas em x = 2 - É sempre positiva ou zero

\textbf{f(x) ≥ 0:} para todo x ∈ ℝ

\textbf{Resposta:} S = ℝ (todos os reais)

\paragraph{Exercício 4}
\label{sec:org4ee6f9e}
(UFMG) Qual a imagem da função f(x) = x² - 4x + 5?

\textbf{Solução:}

a = 1 > 0 → mínimo

xᵥ = 4/2 = 2

yᵥ = f(2) = 4 - 8 + 5 = 1

\textbf{Imagem:} [yᵥ, +∞) = [1, +∞)

\textbf{Resposta:} Im(f) = \{y ∈ ℝ | y ≥ 1\} ou [1, +∞)

\paragraph{Exercício 5}
\label{sec:orgf049039}
Determine para quais valores de k a função f(x) = x² - 6x + k tem valor mínimo igual a 1.

\textbf{Solução:}

a = 1 > 0 → mínimo

yᵥ = 1 (dado)

yᵥ = -Δ/4a 1 = -(b² - 4ac)/4(1) 1 = -(36 - 4k)/4 4 = -(36 - 4k) 4 = -36 + 4k 4k = 40 k = 10

\textbf{Resposta:} k = 10

\textbf{Verificação:} f(x) = x² - 6x + 10 xᵥ = 3 yᵥ = 9 - 18 + 10 = 1 ✓

\subsubsection{Dicas para a Prova}
\label{sec:org8a2b559}
\begin{enumerate}
\item \textbf{Estudo do sinal:} raízes + concavidade
\item \textbf{Δ > 0 e a > 0:} negativa entre as raízes
\item \textbf{Δ > 0 e a < 0:} positiva entre as raízes
\item \textbf{Inequações:} use estudo do sinal
\item \textbf{Máximo:} a < 0, valor = yᵥ
\item \textbf{Mínimo:} a > 0, valor = yᵥ
\item \textbf{≥ ou ≤:} incluir raízes (colchete [])
\item \textbf{> ou <:} excluir raízes (parêntese ())
\end{enumerate}

\subsubsection{Conceitos-Chave para Memorizar}
\label{sec:org9facbc1}
\textbf{Vértice:} - xᵥ = -b/2a - yᵥ = -Δ/4a ou f(xᵥ) - Máximo se a < 0 - Mínimo se a > 0

\textbf{Estudo do Sinal (Δ > 0):} - a > 0: negativa entre raízes - a < 0: positiva entre raízes

\textbf{Imagem:} - a > 0: [yᵥ, +∞) - a < 0: (-∞, yᵥ]

\textbf{Inequações:} 1. Encontrar raízes 2. Esboçar parábola 3. Identificar região

\subsubsection{Fórmulas Essenciais}
\label{sec:orgf466068}
\begin{verbatim}
Vértice:
xᵥ = -b / 2a
yᵥ = -Δ / 4a  ou  yᵥ = f(xᵥ)

Máximo/Mínimo:
a > 0: mínimo = yᵥ
a < 0: máximo = yᵥ

Imagem:
a > 0: Im = [yᵥ, +∞)
a < 0: Im = (-∞, yᵥ]

Estudo do Sinal (Δ > 0, x₁ < x₂):
a > 0:
  f(x) > 0: x < x₁ ou x > x₂
  f(x) < 0: x₁ < x < x₂
  
a < 0:
  f(x) > 0: x₁ < x < x₂
  f(x) < 0: x < x₁ ou x > x₂
\end{verbatim}

\subsubsection{Resumo Visual}
\label{sec:org0133d2b}
\begin{verbatim}
ESTUDO DO SINAL:

Δ > 0, a > 0 (∪):
  + + + + +
  \      /
   \____/
    x₁  x₂
    
+ 0 - - - 0 +
 x₁       x₂

Δ > 0, a < 0 (∩):
    x₁  x₂
   /‾‾‾‾\
  /      \
  - - - - -
  
- 0 + + + 0 -
 x₁       x₂

INEQUAÇÕES:
< ou >: parêntese ( )
≤ ou ≥: colchete [ ]
\end{verbatim}

\noindent\rule{\textwidth}{0.5pt}

\textbf{Tempo de estudo recomendado:} 90 minutos \textbf{Nível de dificuldade:} Médio-Alto \textbf{Importância para a prova:} ⭐⭐⭐⭐⭐ (essencial - muito cobrado!)

\subsection{Aula 31 - Química: Ligações Químicas - Parte 1 (Ligação Iônica) - 30min}
\label{sec:orgd6ab719}
\subsubsection{Introdução: Por que Átomos se Ligam?}
\label{sec:org62cf4a3}
A maioria dos átomos na natureza \textbf{não existe isolada}, mas \textbf{ligada} a outros átomos.

\textbf{Por quê?} - Átomos isolados geralmente são \textbf{instáveis} - Ligações químicas levam à \textbf{estabilidade}

\textbf{Regra do Octeto (ou Teoria do Octeto):} > Átomos tendem a ganhar, perder ou compartilhar elétrons para adquirir \textbf{8 elétrons na camada de valência} (configuração de gás nobre).

\textbf{Exceções:} - \textbf{Hélio (He):} estável com 2 elétrons (camada K completa) - \textbf{Hidrogênio (H):} estável com 2 elétrons (dueto) - Alguns elementos do 3º período podem expandir octeto

\textbf{Gases Nobres:} - Já têm 8 elétrons na valência (He tem 2) - São \textbf{estáveis} e \textbf{inertes} (não fazem ligações facilmente) - Exemplos: He (2), Ne (2,8), Ar (2,8,8)

\subsubsection{Tipos de Ligações Químicas}
\label{sec:org04463fe}
\textbf{Principais tipos:} 1. \textbf{Ligação Iônica} (ou eletrovalente) 2. \textbf{Ligação Covalente} (ou molecular) 3. \textbf{Ligação Metálica}

\textbf{Classificação depende:} - Tipo de átomo envolvido (metal, não-metal, semimetal) - Diferença de eletronegatividade

\subsubsection{Ligação Iônica}
\label{sec:orgf0695f3}
\textbf{Definição:} Ligação entre \textbf{metal e não-metal} através da \textbf{transferência de elétrons}.

\textbf{Características:} - Metal \textbf{perde} elétrons → forma \textbf{cátion} (+) - Não-metal \textbf{ganha} elétrons → forma \textbf{ânion} (-) - Atração eletrostática entre íons de cargas opostas

\textbf{Condições:} - Grande diferença de eletronegatividade (ΔEN ≥ 1,7) - Metal (baixa EN) + Não-metal (alta EN)

\paragraph{Formação da Ligação Iônica}
\label{sec:org6ccd1c9}
\textbf{Exemplo 1: Cloreto de Sódio (NaCl)}

\textbf{Sódio (Na, Z = 11):} - Distribuição: 2, 8, 1 - 1 elétron na valência - Tende a \textbf{perder 1 e⁻} → Na⁺ (2, 8) - estável

\textbf{Cloro (Cl, Z = 17):} - Distribuição: 2, 8, 7 - 7 elétrons na valência - Tende a \textbf{ganhar 1 e⁻} → Cl⁻ (2, 8, 8) - estável

\textbf{Ligação:}

\begin{verbatim}
Na → Na⁺ + e⁻  (perde)
Cl + e⁻ → Cl⁻  (ganha)

Na⁺ + Cl⁻ → NaCl
\end{verbatim}

\textbf{Resultado:} Cristal iônico de NaCl (sal de cozinha)

\textbf{Exemplo 2: Óxido de Magnésio (MgO)}

\textbf{Magnésio (Mg, Z = 12):} - Distribuição: 2, 8, 2 - Perde 2 e⁻ → Mg²⁺ (2, 8)

\textbf{Oxigênio (O, Z = 8):} - Distribuição: 2, 6 - Ganha 2 e⁻ → O²⁻ (2, 8)

\textbf{Ligação:}

\begin{verbatim}
Mg → Mg²⁺ + 2e⁻
O + 2e⁻ → O²⁻

Mg²⁺ + O²⁻ → MgO
\end{verbatim}

\textbf{Exemplo 3: Cloreto de Cálcio (CaCl₂)}

\textbf{Cálcio (Ca, Z = 20):} - Distribuição: 2, 8, 8, 2 - Perde 2 e⁻ → Ca²⁺

\textbf{Cloro (Cl, Z = 17):} - Ganha 1 e⁻ → Cl⁻

\textbf{Problema:} Ca perde 2 e⁻, mas cada Cl ganha apenas 1 e⁻!

\textbf{Solução:} \textbf{2 átomos de Cl} para 1 átomo de Ca

\begin{verbatim}
Ca → Ca²⁺ + 2e⁻
2Cl + 2e⁻ → 2Cl⁻

Ca²⁺ + 2Cl⁻ → CaCl₂
\end{verbatim}

\textbf{Fórmula:} CaCl₂ (1 Ca para 2 Cl)

\subsubsection{Propriedades dos Compostos Iônicos}
\label{sec:orgbe835c4}
\textbf{1. Estado físico:} - \textbf{Sólidos} à temperatura ambiente - Cristais iônicos (arranjo ordenado de íons)

\textbf{2. Ponto de fusão e ebulição:} - \textbf{Altos} (fortes atrações eletrostáticas) - Exemplo: NaCl funde a 801°C

\textbf{3. Condutividade elétrica:} - \textbf{Sólidos:} NÃO conduzem (íons fixos no retículo) - \textbf{Fundidos ou dissolvidos:} CONDUZEM (íons livres)

\textbf{4. Solubilidade:} - Geralmente \textbf{solúveis em água} (solvente polar) - Insolúveis em solventes apolares (gasolina, etc.)

\textbf{5. Dureza:} - Duros, mas \textbf{quebradiços} (fraturam facilmente)

\subsubsection{Fórmula de Compostos Iônicos}
\label{sec:org3472af4}
\textbf{Regra:} Número total de cargas positivas = número total de cargas negativas

\textbf{Método prático (troca de valências):}

Cátion A\^{}(m+) + Ânion B\^{}(n-) → A\_n B\_m

\textbf{Exemplos:}

\begin{enumerate}
\item \textbf{Al³⁺ + O²⁻}
\begin{itemize}
\item Troca: Al₂O₃
\item Óxido de alumínio
\end{itemize}
\item \textbf{Fe³⁺ + S²⁻}
\begin{itemize}
\item Troca: Fe₂S₃
\item Sulfeto de ferro III
\end{itemize}
\item \textbf{Na⁺ + SO₄²⁻}
\begin{itemize}
\item Troca: Na₂SO₄
\item Sulfato de sódio
\end{itemize}
\end{enumerate}

\textbf{Simplificação:} Se houver divisor comum, simplifique.

Exemplo: Ca²⁺ + O²⁻ → Ca₂O₂ → \textbf{CaO} (simplificado)

\subsubsection{Exercícios Resolvidos}
\label{sec:orge527626}
\paragraph{Exercício 1}
\label{sec:org2710b1c}
O que acontece com os átomos de sódio (Na) e cloro (Cl) ao formarem NaCl?

\textbf{Resposta:} - \textbf{Na (2,8,1):} perde 1 elétron → Na⁺ (2,8) - cátion - \textbf{Cl (2,8,7):} ganha 1 elétron → Cl⁻ (2,8,8) - ânion - Formam ligação iônica por atração eletrostática entre Na⁺ e Cl⁻

\paragraph{Exercício 2}
\label{sec:org4441f0f}
Determine a fórmula do composto formado entre alumínio (Al³⁺) e oxigênio (O²⁻).

\textbf{Solução:} Troca de valências: Al³⁺ → índice 2 O²⁻ → índice 3

\textbf{Fórmula:} Al₂O₃

\textbf{Resposta:} Al₂O₃ (óxido de alumínio)

\paragraph{Exercício 3}
\label{sec:orga0d90c8}
Por que o sal de cozinha (NaCl) conduz eletricidade quando dissolvido em água, mas não no estado sólido?

\textbf{Resposta:} - \textbf{Sólido:} íons estão fixos no retículo cristalino, não podem se mover, NÃO conduzem - \textbf{Dissolvido:} íons ficam livres na solução, podem se mover e transportar carga, CONDUZEM eletricidade

\paragraph{Exercício 4}
\label{sec:orgefbda83}
(UFMG) Qual o tipo de ligação entre magnésio (Mg) e cloro (Cl)?

\textbf{Resposta:} \textbf{Ligação iônica}. - Mg é metal (perde elétrons) - Cl é não-metal (ganha elétrons) - Grande diferença de eletronegatividade - Fórmula: MgCl₂

\subsubsection{Dicas para a Prova}
\label{sec:orgc7fb051}
\begin{enumerate}
\item \textbf{Ligação iônica:} metal + não-metal
\item \textbf{Transferência} de elétrons (não compartilhamento)
\item \textbf{Metal perde} → cátion (+)
\item \textbf{Não-metal ganha} → ânion (-)
\item \textbf{Propriedades:} sólidos, altos PF/PE, conduzem fundidos/dissolvidos
\item \textbf{Fórmula:} total de cargas + = total de cargas -
\item \textbf{Regra do octeto:} 8 elétrons na valência (estável)
\item \textbf{Gases nobres:} já estáveis (8 e⁻)
\end{enumerate}

\subsubsection{Conceitos-Chave para Memorizar}
\label{sec:org7515138}
\textbf{Ligação Iônica:} - Metal + Não-metal - Transferência de elétrons - Formação de íons - Atração eletrostática

\textbf{Íons:} - \textbf{Cátion:} íon positivo (perdeu e⁻) - \textbf{Ânion:} íon negativo (ganhou e⁻)

\textbf{Propriedades:} - Sólidos cristalinos - Altos PF e PE - Conduzem fundidos/dissolvidos - Solúveis em água

\textbf{Fórmula:} - Troca de valências - Cargas balanceadas

\subsubsection{Resumo Visual}
\label{sec:orgfecc792}
\begin{verbatim}
LIGAÇÃO IÔNICA:

Metal (baixa EN)  +  Não-metal (alta EN)
        ↓                    ↓
    Perde e⁻            Ganha e⁻
        ↓                    ↓
     Cátion⁺              Ânion⁻
        ↓                    ↓
        └────────┬───────────┘
          Atração eletrostática
                 ↓
          Composto Iônico

Exemplo:
Na (2,8,1) → Na⁺ (2,8) + e⁻
Cl (2,8,7) + e⁻ → Cl⁻ (2,8,8)

Na⁺ + Cl⁻ → NaCl

Propriedades:
├─ Sólidos (cristais)
├─ Altos PF/PE
├─ Conduzem fundidos/dissolvidos
└─ Solúveis em H₂O
\end{verbatim}

\noindent\rule{\textwidth}{0.5pt}

\textbf{Tempo de estudo recomendado:} 30 minutos \textbf{Nível de dificuldade:} Médio \textbf{Importância para a prova:} ⭐⭐⭐⭐ (muito importante - ligações são fundamentais!)

\section{11/27 - Férias Dia 2}
\label{sec:org3c78934}
\subsection{Aula 32 - Matemática: Função Quadrática - Parte 3 (Inequações e Sistemas) - 90min}
\label{sec:org52bd402}
\subsubsection{Revisão: Função Quadrática}
\label{sec:org88de50f}
Já estudamos: - \textbf{Parte 1 (Aula 26):} Definição, gráfico, raízes (Bhaskara), vértice - \textbf{Parte 2 (Aula 30):} Zeros, máximo/mínimo, estudo do sinal básico

\textbf{Nesta aula:} Inequações mais complexas e sistemas envolvendo função quadrática.

\subsubsection{Inequações do 2º Grau - Aprofundamento}
\label{sec:org0a16d32}
\paragraph{Inequações-Produto}
\label{sec:org72abb40}
\textbf{Formato:} (ax² + bx + c)(dx² + ex + f) > 0 (ou <, ≥, ≤)

\textbf{Método:} 1. Estudar o sinal de cada fator separadamente 2. Fazer quadro de sinais 3. Multiplicar os sinais 4. Responder conforme pedido

\textbf{Exemplo:} (x² - 4)(x² - 9) ≤ 0

\textbf{Passo 1: Estudar cada fator}

\textbf{Fator 1:} x² - 4 - Raízes: x = ±2 - a = 1 > 0 (∪) - Sinal: + | - | + -2 2

\textbf{Fator 2:} x² - 9 - Raízes: x = ±3 - a = 1 > 0 (∪) - Sinal: + | - | + -3 3

\textbf{Passo 2: Quadro de sinais}

\begin{verbatim}
        -3   -2    2    3
x²-4:   +    +  0  -  0  +  +
x²-9:   +  0 -    -    -  0  +
────────────────────────────────
Produto: + 0 -  0  +  0  - 0  +
\end{verbatim}

\textbf{Passo 3: Produto ≤ 0} (negativo ou zero)

Intervalos: [-3, -2] ∪ [2, 3]

\textbf{Resposta:} S = [-3, -2] ∪ [2, 3]

\paragraph{Inequações-Quociente}
\label{sec:org30f07c9}
\textbf{Formato:} (ax² + bx + c)/(dx² + ex + f) > 0

\textbf{Método:} 1. Estudar sinal do numerador 2. Estudar sinal do denominador 3. Quadro de sinais (dividir sinais) 4. \textbf{IMPORTANTE:} Denominador ≠ 0 (excluir raízes do denominador)

\textbf{Exemplo:} (x² - 1)/(x² - 4) ≥ 0

\textbf{Numerador:} x² - 1 - Raízes: x = ±1 - a = 1 > 0 - Sinal: + | - | + -1 1

\textbf{Denominador:} x² - 4 - Raízes: x = ±2 (EXCLUIR da resposta!) - a = 1 > 0 - Sinal: + | - | + -2 2

\textbf{Quadro:}

\begin{verbatim}
          -2   -1    1    2
Numer.:   +    + 0  - 0  +    +
Denom.:   + ≠0 -    -    - ≠0 +
────────────────────────────────
Quoc.:    +  ∅ - 0  + 0  -  ∅  +
\end{verbatim}

\textbf{Quociente ≥ 0:} positivo ou zero

\textbf{Resposta:} S = (-∞, -2) ∪ [-1, 1] ∪ (2, +∞)

\textbf{Observe:} -2 e 2 EXCLUÍDOS (denominador zero), -1 e 1 INCLUÍDOS (numerador zero)

\paragraph{Inequações Simultâneas}
\label{sec:org1643468}
\textbf{Formato:} Sistema de inequações

\textbf{Exemplo:}

\begin{verbatim}
{ x² - 5x + 6 < 0
{ x² - 4 ≥ 0
\end{verbatim}

\textbf{Resolver cada uma:}

\textbf{1ª inequação:} x² - 5x + 6 < 0 - Raízes: 2 e 3 - a > 0 → negativa entre raízes - S₁ = (2, 3)

\textbf{2ª inequação:} x² - 4 ≥ 0 - Raízes: -2 e 2 - a > 0 → positiva fora raízes (ou nas raízes) - S₂ = (-∞, -2] ∪ [2, +∞)

\textbf{Interseção S₁ ∩ S₂:}

\begin{verbatim}
S₁:    ═══════(───)═══════
              2   3
              
S₂: ══════]       [════════
         -2       2
         
S₁∩S₂:          [─)
                2 3
\end{verbatim}

\textbf{Resposta:} S = [2, 3)

\subsubsection{Sistemas de Equações do 2º Grau}
\label{sec:org0a7149e}
\paragraph{Sistema Linear-Quadrático}
\label{sec:orge956bc6}
\textbf{Formato:}

\begin{verbatim}
{ y = ax + b         (reta)
{ y = cx² + dx + e   (parábola)
\end{verbatim}

\textbf{Método:} Substituição

\textbf{Exemplo:}

\begin{verbatim}
{ y = 2x + 1
{ y = x² - 2x + 3
\end{verbatim}

\textbf{Substituição:} 2x + 1 = x² - 2x + 3 0 = x² - 4x + 2 x² - 4x + 2 = 0

\textbf{Bhaskara:} Δ = 16 - 8 = 8 x = (4 ± √8)/2 = (4 ± 2√2)/2 = 2 ± √2

x₁ = 2 + √2 x₂ = 2 - √2

\textbf{Encontrar y:} y₁ = 2(2 + √2) + 1 = 5 + 2√2 y₂ = 2(2 - √2) + 1 = 5 - 2√2

\textbf{Soluções:} - (2 + √2, 5 + 2√2) - (2 - √2, 5 - 2√2)

\textbf{Interpretação geométrica:} Pontos onde a reta intercepta a parábola.

\paragraph{Sistema Quadrático-Quadrático}
\label{sec:orgb51a7b1}
\textbf{Formato:}

\begin{verbatim}
{ y = ax² + bx + c
{ y = dx² + ex + f
\end{verbatim}

\textbf{Exemplo:}

\begin{verbatim}
{ y = x² - 4
{ y = -x² + 2x + 4
\end{verbatim}

\textbf{Igualando:} x² - 4 = -x² + 2x + 4 2x² - 2x - 8 = 0 x² - x - 4 = 0

Δ = 1 + 16 = 17 x = (1 ± √17)/2

E assim por diante\ldots{}

\subsubsection{Aplicações - Problemas Contextualizados}
\label{sec:orgac69f27}
\paragraph{Problema 1: Geometria}
\label{sec:orgd18b9eb}
Um retângulo tem perímetro 20 cm e área 24 cm². Quais suas dimensões?

\textbf{Solução:}

Sejam x e y os lados.

\begin{verbatim}
{ 2x + 2y = 20  →  x + y = 10  →  y = 10 - x
{ xy = 24
\end{verbatim}

Substituindo: x(10 - x) = 24 10x - x² = 24 x² - 10x + 24 = 0

Fatorando: (x - 4)(x - 6) = 0 x = 4 ou x = 6

Se x = 4: y = 6 Se x = 6: y = 4

\textbf{Resposta:} Dimensões: 4 cm × 6 cm

\paragraph{Problema 2: Movimento (Física)}
\label{sec:orgbfaf2ba}
Um projétil é lançado verticalmente. Sua altura h(t) em metros no tempo t (segundos) é dada por:

h(t) = -5t² + 20t + 5

\begin{enumerate}
\item Qual a altura máxima?
\item Quando atinge o solo (h = 0)?
\end{enumerate}

\textbf{Solução:}

\begin{enumerate}
\item \textbf{Altura máxima = yᵥ}
\end{enumerate}

a = -5, b = 20, c = 5

tᵥ = -20/2(-5) = 20/10 = 2 s

hᵥ = -5(4) + 20(2) + 5 = -20 + 40 + 5 = 25 m

\textbf{Máxima: 25 m} (em t = 2 s)

\begin{enumerate}
\setcounter{enumi}{1}
\item \textbf{h = 0:}
\end{enumerate}

-5t² + 20t + 5 = 0 t² - 4t - 1 = 0

Δ = 16 + 4 = 20

t = (4 ± √20)/2 = (4 ± 2√5)/2 = 2 ± √5

t₁ = 2 - √5 ≈ -0,24 s (descartado: negativo) t₂ = 2 + √5 ≈ 4,24 s ✓

\textbf{Resposta:} Aproximadamente 4,24 s

\paragraph{Problema 3: Economia}
\label{sec:org5f814ca}
Uma empresa descobriu que o lucro L (em mil reais) ao vender x unidades é dado por:

L(x) = -x² + 40x - 300

\begin{enumerate}
\item Quantas unidades deve vender para lucro máximo?
\item Qual o lucro máximo?
\item Para quais quantidades há prejuízo (L < 0)?
\end{enumerate}

\textbf{Solução:}

a = -1, b = 40, c = -300

\begin{enumerate}
\item \textbf{xᵥ = -40/2(-1) = 20 unidades}

\item \textbf{Lᵥ = -(20)² + 40(20) - 300} = -400 + 800 - 300 = 100
\end{enumerate}

\textbf{Lucro máximo: 100 mil reais}

\begin{enumerate}
\setcounter{enumi}{2}
\item \textbf{L < 0:}
\end{enumerate}

Raízes: Δ = 1600 - 1200 = 400 x = (40 ± 20)/2 x₁ = 10, x₂ = 30

a = -1 < 0 (∩): negativa fora das raízes

\textbf{Prejuízo:} x < 10 ou x > 30

Mas x ≥ 0 (não há quantidade negativa):

\textbf{Resposta:} 0 ≤ x < 10 ou x > 30 unidades

\subsubsection{Exercícios Resolvidos}
\label{sec:orga034066}
\paragraph{Exercício 1}
\label{sec:orgb61d277}
Resolva: (x - 2)(x² - 9) > 0

\textbf{Solução:}

\textbf{Fator 1:} x - 2 - Raiz: x = 2 - Sinal: - | + 2

\textbf{Fator 2:} x² - 9 - Raízes: x = ±3 - Sinal: + | - | + -3 3

\textbf{Quadro:}

\begin{verbatim}
      -3    2    3
x-2:  -  -  -  0 +  +
x²-9: + 0 -    - 0 +
──────────────────────
Prod: - 0 +  0 - 0 +
\end{verbatim}

\textbf{Produto > 0:} (-3, 2) ∪ (3, +∞)

\textbf{Resposta:} S = (-3, 2) ∪ (3, +∞)

\paragraph{Exercício 2}
\label{sec:org6297d50}
Resolva: (x² - 4)/(x - 3) ≤ 0

\textbf{Solução:}

\textbf{Numerador:} x² - 4 - Raízes: ±2 - Sinal: + | - | + -2 2

\textbf{Denominador:} x - 3 - Raiz: 3 (EXCLUIR) - Sinal: - | + 3

\textbf{Quadro:}

\begin{verbatim}
      -2    2    3
Num.: + 0 - 0 +    +
Den.: -   -   - ≠0 +
──────────────────────
Quo.: - 0 + 0 -  ∅  +
\end{verbatim}

\textbf{Quociente ≤ 0:} [-2, 2] ∪ (2, 3)

Simplificando: [-2, 2] ∪ (2, 3) = [-2, 3)

\textbf{Resposta:} S = [-2, 3)

(3 excluído pois anula denominador)

\paragraph{Exercício 3}
\label{sec:org180d482}
Determine os pontos de interseção de y = x² e y = 2x + 3.

\textbf{Solução:}

x² = 2x + 3 x² - 2x - 3 = 0 (x - 3)(x + 1) = 0 x = 3 ou x = -1

\textbf{Para x = 3:} y = 9 \textbf{Para x = -1:} y = 1

\textbf{Resposta:} Pontos (3, 9) e (-1, 1)

\subsubsection{Dicas para a Prova}
\label{sec:org74485ed}
\begin{enumerate}
\item \textbf{Inequação-produto:} quadro de sinais e multiplicar
\item \textbf{Inequação-quociente:} EXCLUIR raízes do denominador
\item \textbf{≥ ou ≤:} incluir raízes do numerador (não do denominador!)
\item \textbf{Sistema simultâneo:} interseção das soluções
\item \textbf{Problema contextualizado:} montar função a partir do enunciado
\item \textbf{Máximo/mínimo:} sempre vértice (yᵥ)
\item \textbf{Geometria:} cuidado com dimensões positivas
\item \textbf{Física:} tempo negativo não tem sentido (descartar)
\end{enumerate}

\subsubsection{Conceitos-Chave para Memorizar}
\label{sec:org423c662}
\textbf{Inequação-Produto:} - Quadro de sinais - Multiplicar sinais dos fatores

\textbf{Inequação-Quociente:} - Dividir sinais - Denominador ≠ 0 (excluir da resposta)

\textbf{Sistema Linear-Quadrático:} - Substituição - 0, 1 ou 2 soluções

\textbf{Problemas:} - Identificar variáveis - Montar equação/sistema - Resolver - Interpretar (descartar soluções sem sentido físico)

\subsubsection{Fórmulas Essenciais}
\label{sec:org747db53}
\begin{verbatim}
Quadro de Sinais:
1. Encontrar raízes de cada fator
2. Determinar sinal de cada fator
3. Combinar sinais (× para produto, ÷ para quociente)
4. Ler região pedida

Inequação-Quociente:
Numerador = 0: pode incluir (se ≥ ou ≤)
Denominador = 0: SEMPRE excluir (∅)

Sistema de Inequações:
Resolver cada uma → Interseção S₁ ∩ S₂

Aplicações:
- Perímetro retângulo: 2(x + y)
- Área retângulo: xy
- Altura projétil: h(t) = -gt²/2 + v₀t + h₀
\end{verbatim}

\subsubsection{Resumo Visual}
\label{sec:orgd1bdfe2}
\begin{verbatim}
INEQUAÇÃO-PRODUTO: (f₁)(f₂) > 0

        raízes f₁  raízes f₂
f₁:     ─────0────0─────
f₂:     ──0─────────0───
        ───────────────
Prod:   [combinar sinais]

INEQUAÇÃO-QUOCIENTE: f₁/f₂ ≥ 0

Num:    ────0────0────  (pode incluir)
Den:    ──≠0────────≠0  (EXCLUIR sempre)
        ──────────────
Quo:    [dividir sinais]

SISTEMA:
S₁:  ════(────)════
S₂:  ══════[──────)══

S₁∩S₂:    [──)  (interseção)
\end{verbatim}

\noindent\rule{\textwidth}{0.5pt}

\textbf{Tempo de estudo recomendado:} 90 minutos \textbf{Nível de dificuldade:} Alto \textbf{Importância para a prova:} ⭐⭐⭐⭐⭐ (essencial - inequações sempre caem!)

\subsection{Aula 33 - Química: Ligações Químicas - Parte 2 (Ligação Covalente) - 30min}
\label{sec:org81f3172}
\subsubsection{Revisão: Ligações Químicas}
\label{sec:org996bc3e}
Na Aula 31, estudamos: - \textbf{Ligação Iônica:} metal + não-metal, transferência de elétrons, íons

\textbf{Nesta aula:} Ligação Covalente (molecular)

\subsubsection{Ligação Covalente}
\label{sec:org3f798a0}
\textbf{Definição:} Ligação entre \textbf{não-metais} através do \textbf{compartilhamento de elétrons}.

\textbf{Características:} - Não-metal + Não-metal - \textbf{Compartilham} elétrons (não transferem) - Formam \textbf{moléculas} - Pequena diferença de eletronegatividade (ΔEN < 1,7)

\textbf{Objetivo:} Ambos os átomos atingem configuração de gás nobre (regra do octeto ou dueto).

\subsubsection{Formação da Ligação Covalente}
\label{sec:org88fe3c8}
\paragraph{Exemplo 1: Hidrogênio (H₂)}
\label{sec:orgd75733f}
\textbf{H (Z = 1):} - Distribuição: 1 - Precisa de 1 elétron para completar (dueto: 2 elétrons)

\textbf{Ligação:}

\begin{verbatim}
H • + • H  →  H : H  ou  H─H
\end{verbatim}

Cada H compartilha 1 elétron → ambos ficam com 2 elétrons (estáveis)

\textbf{Fórmula molecular:} H₂

\textbf{Fórmula estrutural:} H─H (traço = par de elétrons compartilhado)

\paragraph{Exemplo 2: Cloro (Cl₂)}
\label{sec:org5e272f8}
\textbf{Cl (Z = 17):} - Distribuição: 2, 8, 7 - Precisa de 1 elétron para completar octeto (8)

\textbf{Representação de Lewis:}

\begin{verbatim}
:Cl· + ·Cl:  →  :Cl:Cl:  ou  Cl─Cl
\end{verbatim}

Cada Cl compartilha 1 elétron → ambos com 8 elétrons na valência

\textbf{Fórmula molecular:} Cl₂

\paragraph{Exemplo 3: Água (H₂O)}
\label{sec:org6d24b1c}
\textbf{O (Z = 8):} - Distribuição: 2, 6 - Precisa de 2 elétrons

\textbf{H:} - Precisa de 1 elétron (cada)

\textbf{Ligação:}

\begin{verbatim}
    ·O·
   H   H
   
Cada H compartilha 1 elétron com O
O compartilha 2 elétrons (1 com cada H)
\end{verbatim}

\textbf{Fórmula estrutural:}

\begin{verbatim}
H─O─H
\end{verbatim}

O tem 8 elétrons (2 ligações + 4 não-ligantes) Cada H tem 2 elétrons

\paragraph{Exemplo 4: Amônia (NH₃)}
\label{sec:org1ffa1f5}
\textbf{N (Z = 7):} - Distribuição: 2, 5 - Precisa de 3 elétrons

\textbf{H:} precisa de 1 (cada um)

\textbf{Fórmula estrutural:}

\begin{verbatim}
  H
  │
H─N─H
\end{verbatim}

N faz 3 ligações (compartilha 6 elétrons) + 2 não-ligantes = 8 total

\paragraph{Exemplo 5: Metano (CH₄)}
\label{sec:orga8308d7}
\textbf{C (Z = 6):} - Distribuição: 2, 4 - Precisa de 4 elétrons

\textbf{Fórmula estrutural:}

\begin{verbatim}
  H
  │
H─C─H
  │
  H
\end{verbatim}

C faz 4 ligações → 8 elétrons (octeto completo)

\subsubsection{Tipos de Ligações Covalentes}
\label{sec:org6cdbb7c}
\paragraph{1. Ligação Simples}
\label{sec:orgd2c9bd2}
Um par de elétrons compartilhado.

\textbf{Representação:} A─B

\textbf{Exemplos:} - H─H - Cl─Cl - H─O─H

\paragraph{2. Ligação Dupla}
\label{sec:orgb25e9b5}
Dois pares de elétrons compartilhados.

\textbf{Representação:} A═B

\textbf{Exemplo: Oxigênio (O₂)}

\textbf{O:} 2, 6 (precisa de 2 elétrons)

\begin{verbatim}
:O::O:  ou  O═O
\end{verbatim}

Cada O compartilha 2 elétrons (4 no total) → ligação dupla

\paragraph{3. Ligação Tripla}
\label{sec:org23429df}
Três pares de elétrons compartilhados.

\textbf{Representação:} A≡B

\textbf{Exemplo: Nitrogênio (N₂)}

\textbf{N:} 2, 5 (precisa de 3 elétrons)

\begin{verbatim}
:N:::N:  ou  N≡N
\end{verbatim}

Cada N compartilha 3 elétrons (6 no total) → ligação tripla

\textbf{Outro exemplo: Gás carbônico (CO₂)}

\begin{verbatim}
O═C═O
\end{verbatim}

C faz duas ligações duplas (4 pares = 8 elétrons)

\subsubsection{Polaridade das Ligações Covalentes}
\label{sec:orgcb37ee4}
\paragraph{Ligação Covalente Apolar}
\label{sec:org9a8a5df}
\textbf{Condição:} ΔEN = 0 (átomos iguais ou eletronegatividades iguais)

\textbf{Características:} - Elétrons compartilhados \textbf{igualmente} - Sem polo positivo/negativo

\textbf{Exemplos:} - H₂, O₂, N₂, Cl₂ (moléculas diatômicas iguais) - CH₄ (simetria molecular)

\paragraph{Ligação Covalente Polar}
\label{sec:org9df5b5d}
\textbf{Condição:} 0 < ΔEN < 1,7 (átomos diferentes)

\textbf{Características:} - Elétrons compartilhados \textbf{desigualmente} - Átomo mais eletronegativo “puxa” mais os elétrons - Forma \textbf{polos:} δ⁺ (parcialmente positivo) e δ⁻ (parcialmente negativo)

\textbf{Exemplos:} - \textbf{HCl:} Cl mais eletronegativo → H\^{}\{(δ+)─Cl\}(δ-) - \textbf{H₂O:} O mais eletronegativo → H\^{}\{(δ+)─O\}(δ-)─H\^{}(δ+)

\textbf{Dipolo elétrico:} representado por seta → apontando para polo negativo

\begin{verbatim}
H → Cl
  ↑
 dipolo
\end{verbatim}

\subsubsection{Geometria Molecular (Introdução)}
\label{sec:org56059dc}
A \textbf{forma tridimensional} da molécula influencia propriedades.

\textbf{Exemplos:}

\textbf{Linear:} CO₂ (O═C═O) - 180°

\textbf{Angular:} H₂O

\begin{verbatim}
H
 \
  O  (104,5°)
 /
H
\end{verbatim}

\textbf{Trigonal plana:} BF₃ - 120°

\textbf{Tetraédrica:} CH₄

\begin{verbatim}
  H
  │
H─C─H  (109,5°)
  │
  H
\end{verbatim}

\textbf{Piramidal:} NH₃

(Aprofundaremos geometria em aulas futuras)

\subsubsection{Propriedades dos Compostos Covalentes (Moleculares)}
\label{sec:orged9f626}
\textbf{1. Estado físico:} - \textbf{Gases} ou \textbf{líquidos} à temperatura ambiente (maioria) - Alguns sólidos (açúcar, gelo)

\textbf{2. Ponto de fusão e ebulição:} - \textbf{Baixos} (forças intermoleculares fracas) - Exceção: macromoléculas (diamante, quartzo)

\textbf{3. Condutividade elétrica:} - \textbf{NÃO conduzem} (não têm íons livres) - Exceção: grafite (estrutura especial)

\textbf{4. Solubilidade:} - Polares: solúveis em solventes polares (água) - Apolares: solúveis em solventes apolares (gasolina) - “Semelhante dissolve semelhante”

\subsubsection{Comparação: Ligação Iônica vs. Covalente}
\label{sec:orgaf51f9e}
\begin{verbatim}
┌──────────────┬──────────────┬──────────────┐
│              │    IÔNICA    │  COVALENTE   │
├──────────────┼──────────────┼──────────────┤
│ Átomos       │Metal+Não-met.│Não-metal+Não │
│ Processo     │ Transferência│Compartilhamento│
│ Formam       │    Íons      │  Moléculas   │
│ Estado físico│   Sólidos    │ Gases/Líquidos│
│ PF/PE        │    Altos     │    Baixos    │
│ Condutividade│Fund/Dissol.  │  Não conduz  │
│ Exemplo      │    NaCl      │     H₂O      │
└──────────────┴──────────────┴──────────────┘
\end{verbatim}

\subsubsection{Exercícios Resolvidos}
\label{sec:orge50aceb}
\paragraph{Exercício 1}
\label{sec:org30100e8}
Represente a fórmula estrutural do F₂ (Z do F = 9).

\textbf{Solução:}

F: 2, 7 (precisa de 1 elétron)

\begin{verbatim}
:F: + :F:  →  :F:F:  ou  F─F
\end{verbatim}

\textbf{Resposta:} F─F (ligação simples)

\paragraph{Exercício 2}
\label{sec:orgde9e27a}
Quantas ligações o carbono faz na molécula CO₂?

\textbf{Solução:}

C: 2, 4 (precisa de 4 elétrons = 4 ligações)

Estrutura: O═C═O

C faz \textbf{2 ligações duplas} (total: 4 ligações)

\textbf{Resposta:} 2 ligações duplas

\paragraph{Exercício 3}
\label{sec:org8b14ef0}
A ligação H─Cl é polar ou apolar? Por quê?

\textbf{Solução:}

H e Cl são átomos \textbf{diferentes} Cl é \textbf{mais eletronegativo} que H ΔEN > 0

\textbf{Resposta:} \textbf{Polar}. Cl puxa mais os elétrons, formando H\^{}\{(δ+)─Cl\}(δ-).

\paragraph{Exercício 4}
\label{sec:org26f0b7d}
(UFMG) Qual o tipo de ligação em cada substância: a) NaCl b) H₂ c) CaO

\textbf{Solução:}

\begin{enumerate}
\item \textbf{NaCl:} Na (metal) + Cl (não-metal) → \textbf{Iônica}

\item \textbf{H₂:} H + H (não-metais iguais) → \textbf{Covalente apolar}

\item \textbf{CaO:} Ca (metal) + O (não-metal) → \textbf{Iônica}
\end{enumerate}

\subsubsection{Dicas para a Prova}
\label{sec:org8e6e7fb}
\begin{enumerate}
\item \textbf{Ligação covalente:} não-metal + não-metal
\item \textbf{Compartilhamento} de elétrons (não transferência)
\item \textbf{Ligação simples:} 1 par (─)
\item \textbf{Ligação dupla:} 2 pares (═)
\item \textbf{Ligação tripla:} 3 pares (≡)
\item \textbf{Apolar:} ΔEN = 0 (átomos iguais ou simétricos)
\item \textbf{Polar:} 0 < ΔEN < 1,7 (átomos diferentes)
\item \textbf{Propriedades:} baixos PF/PE, não conduzem
\end{enumerate}

\subsubsection{Conceitos-Chave para Memorizar}
\label{sec:org17a1ce8}
\textbf{Ligação Covalente:} - Não-metal + Não-metal - Compartilhamento de elétrons - Formam moléculas

\textbf{Tipos:} - Simples: ─ (1 par) - Dupla: ═ (2 pares) - Tripla: ≡ (3 pares)

\textbf{Polaridade:} - Apolar: ΔEN = 0 - Polar: 0 < ΔEN < 1,7

\textbf{Propriedades (molecular):} - Baixos PF/PE - Gases ou líquidos - Não conduzem eletricidade

\subsubsection{Resumo Visual}
\label{sec:org2b4002d}
\begin{verbatim}
LIGAÇÃO COVALENTE:

Não-metal  +  Não-metal
     ↓            ↓
 Compartilham elétrons
          ↓
      Moléculas

Exemplos:
H · + · H  →  H:H  →  H─H  (H₂)

:Cl· + ·Cl: → :Cl:Cl: → Cl─Cl (Cl₂)

·Ö· + H + H → H:Ö:H → H─O─H (H₂O)
              

Tipos de Ligação:
Simples:  A─B   (1 par)
Dupla:    A═B   (2 pares)
Tripla:   A≡B   (3 pares)

Polaridade:
Apolar:  H─H  (ΔEN = 0)
         Cl─Cl

Polar:   H^δ+─Cl^δ-  (ΔEN > 0)
         H^δ+─O^δ-─H^δ+
\end{verbatim}

\noindent\rule{\textwidth}{0.5pt}

\textbf{Tempo de estudo recomendado:} 30 minutos \textbf{Nível de dificuldade:} Médio \textbf{Importância para a prova:} ⭐⭐⭐⭐⭐ (essencial - ligação covalente é fundamental!)

\section{11/28 - Férias Dia 3}
\label{sec:org2330076}
\subsection{Aula 34 - Matemática: Função Exponencial - Parte 1 (Potenciação e Propriedades) - 90min}
\label{sec:orgda2eb3d}
\subsubsection{Introdução: Crescimento Exponencial}
\label{sec:org866542f}
\textbf{Problemas que envolvem crescimento/decrescimento muito rápido:} - População de bactérias (duplica a cada hora) - Juros compostos (dinheiro cresce exponencialmente) - Desintegração radioativa (decaimento exponencial)

Esses fenômenos são modelados por \textbf{funções exponenciais}.

\subsubsection{Revisão: Potenciação}
\label{sec:org97745c1}
\paragraph{Definição}
\label{sec:org6522260}
\textbf{Potência:} a\^{}n (a elevado a n) - \textbf{Base:} a - \textbf{Expoente:} n - \textbf{Resultado:} potência

\textbf{Exemplos:} - 2³ = 2 × 2 × 2 = 8 - 5² = 5 × 5 = 25 - 10⁴ = 10.000

\paragraph{Casos Especiais}
\label{sec:org128c1b4}
\textbf{Expoente zero:}

\begin{verbatim}
a⁰ = 1  (para a ≠ 0)
\end{verbatim}

Exemplos: 5⁰ = 1, 100⁰ = 1

\textbf{Expoente um:}

\begin{verbatim}
a¹ = a
\end{verbatim}

\textbf{Expoente negativo:}

\begin{verbatim}
a^(-n) = 1/a^n
\end{verbatim}

Exemplos: - 2\^{}(-3) = 1/2³ = 1/8 - 5\^{}(-2) = 1/5² = 1/25

\textbf{Expoente fracionário:}

\begin{verbatim}
a^(m/n) = ⁿ√(a^m)
\end{verbatim}

Exemplos: - 4\^{}(1/2) = √4 = 2 - 8\^{}(2/3) = ³√(8²) = ³√64 = 4 - 27\^{}(1/3) = ³√27 = 3

\subsubsection{Propriedades da Potenciação}
\label{sec:org11dc078}
\paragraph{1. Multiplicação de Mesma Base}
\label{sec:orge850989}
\begin{verbatim}
a^m × a^n = a^(m+n)
\end{verbatim}

\textbf{Exemplos:} - 2³ × 2² = 2\^{}(3+2) = 2⁵ = 32 - x⁴ × x³ = x⁷

\textbf{Regra:} Mantém a base, soma os expoentes.

\paragraph{2. Divisão de Mesma Base}
\label{sec:org8e46873}
\begin{verbatim}
a^m / a^n = a^(m-n)
\end{verbatim}

\textbf{Exemplos:} - 5⁷ / 5⁴ = 5\^{}(7-4) = 5³ = 125 - x⁸ / x³ = x⁵

\textbf{Regra:} Mantém a base, subtrai os expoentes.

\paragraph{3. Potência de Potência}
\label{sec:org87f640f}
\begin{verbatim}
(a^m)^n = a^(m×n)
\end{verbatim}

\textbf{Exemplos:} - (2³)² = 2\^{}(3×2) = 2⁶ = 64 - (x²)⁵ = x\^{}(2×5) = x¹⁰

\textbf{Regra:} Mantém a base, multiplica os expoentes.

\paragraph{4. Potência de Produto}
\label{sec:org35bc66d}
\begin{verbatim}
(a × b)^n = a^n × b^n
\end{verbatim}

\textbf{Exemplos:} - (2 × 3)² = 2² × 3² = 4 × 9 = 36 - (xy)³ = x³y³

\paragraph{5. Potência de Quociente}
\label{sec:org4e1c458}
\begin{verbatim}
(a/b)^n = a^n / b^n
\end{verbatim}

\textbf{Exemplos:} - (2/3)² = 2²/3² = 4/9 - (x/y)⁴ = x⁴/y⁴

\subsubsection{Equações Exponenciais Simples}
\label{sec:orgf75f0be}
\textbf{Equação exponencial:} incógnita no expoente.

\textbf{Formato:} a\^{}x = b

\paragraph{Método 1: Bases Iguais}
\label{sec:orgf0b735f}
Se conseguirmos escrever ambos os lados com a mesma base:

\begin{verbatim}
a^x = a^y  →  x = y
\end{verbatim}

\textbf{Exemplo 1:} 2\^{}x = 8

8 = 2³

2\^{}x = 2³

\textbf{x = 3}

\textbf{Exemplo 2:} 5\^{}(x+1) = 25

25 = 5²

5\^{}(x+1) = 5²

x + 1 = 2

\textbf{x = 1}

\textbf{Exemplo 3:} 3\^{}(2x) = 1/9

1/9 = 1/3² = 3\^{}(-2)

3\^{}(2x) = 3\^{}(-2)

2x = -2

\textbf{x = -1}

\textbf{Exemplo 4:} 4\^{}x = 32

Bases diferentes, mas podem ser escritas como potências de 2:

4 = 2² 32 = 2⁵

(2²)\^{}x = 2⁵

2\^{}(2x) = 2⁵

2x = 5

\textbf{x = 5/2}

\paragraph{Casos com Soma/Produto no Expoente}
\label{sec:org32687e2}
\textbf{Exemplo 5:} 2\^{}(x+2) = 2\^{}x + 12

Não podemos igualar expoentes diretamente!

\textbf{Estratégia:} Fatorar

2\^{}(x+2) = 2\^{}x × 2² = 4 × 2\^{}x

4 × 2\^{}x = 2\^{}x + 12

4 × 2\^{}x - 2\^{}x = 12

3 × 2\^{}x = 12

2\^{}x = 4

2\^{}x = 2²

\textbf{x = 2}

\subsubsection{Inequações Exponenciais Simples}
\label{sec:org89f56ec}
\textbf{Formato:} a\^{}x > b (ou <, ≥, ≤)

\textbf{Regra depende da base:}

\textbf{Se a > 1:} função crescente - a\^{}x > a\^{}y → x > y

\textbf{Se 0 < a < 1:} função decrescente - a\^{}x > a\^{}y → x < y (inverte!)

\textbf{Exemplo 1:} 2\^{}x > 8

2\^{}x > 2³

Base 2 > 1 (crescente) → mantém sinal

\textbf{x > 3}

\textbf{Exemplo 2:} (1/2)\^{}x ≥ 4

(1/2)\^{}x ≥ (1/2)\^{}(-2)

Base 1/2 < 1 (decrescente) → \textbf{inverte} sinal

\textbf{x ≤ -2}

\subsubsection{Gráfico da Função Exponencial (Introdução)}
\label{sec:org14f10d3}
\textbf{Função exponencial:} f(x) = a\^{}x (a > 0, a ≠ 1)

\paragraph{Caso 1: a > 1 (Crescente)}
\label{sec:org2330be7}
\textbf{Exemplo:} f(x) = 2\^{}x

\begin{verbatim}
x  | f(x)
-2 | 1/4
-1 | 1/2
 0 | 1
 1 | 2
 2 | 4
 3 | 8
\end{verbatim}

\textbf{Gráfico:}

\begin{verbatim}
    |
  8 |         •
  4 |       •
  2 |     •
  1 |   •─────────
1/2 | •
    |______________
    -2  0  2
\end{verbatim}

\textbf{Características:} - \textbf{Crescente} - Passa por (0, 1) - Eixo x é \textbf{assíntota horizontal} (nunca toca) - Sempre positiva (f(x) > 0)

\paragraph{Caso 2: 0 < a < 1 (Decrescente)}
\label{sec:orge12d7d7}
\textbf{Exemplo:} f(x) = (1/2)\^{}x

\begin{verbatim}
x  | f(x)
-2 | 4
-1 | 2
 0 | 1
 1 | 1/2
 2 | 1/4
\end{verbatim}

\textbf{Gráfico:}

\begin{verbatim}
    |
  4 | •
  2 |  •
  1 |───•─────
1/2 |     •
    |       •
    |______________
    -2  0  2
\end{verbatim}

\textbf{Características:} - \textbf{Decrescente} - Passa por (0, 1) - Eixo x é assíntota - Sempre positiva

\subsubsection{Exercícios Resolvidos}
\label{sec:org3588c88}
\paragraph{Exercício 1}
\label{sec:org1801d45}
Calcule: a) 2⁵ b) 3\^{}(-2) c) 16\^{}(1/2)

\textbf{Soluções:}

\begin{enumerate}
\item 2⁵ = 2 × 2 × 2 × 2 × 2 = 32

\item 3\^{}(-2) = 1/3² = 1/9

\item 16\^{}(1/2) = √16 = 4
\end{enumerate}

\textbf{Respostas:} a) 32; b) 1/9; c) 4

\paragraph{Exercício 2}
\label{sec:orgcd6d4b2}
Simplifique: 2⁷ × 2³ / 2⁴

\textbf{Solução:}

= 2\^{}(7+3) / 2⁴ = 2¹⁰ / 2⁴ = 2\^{}(10-4) = 2⁶ = 64

\textbf{Resposta:} 64

\paragraph{Exercício 3}
\label{sec:org8ea1da9}
Resolva: 3\^{}x = 81

\textbf{Solução:}

81 = 3⁴

3\^{}x = 3⁴

\textbf{x = 4}

\paragraph{Exercício 4}
\label{sec:orgb16aec6}
Resolva: 2\^{}(x-1) = 1/8

\textbf{Solução:}

1/8 = 1/2³ = 2\^{}(-3)

2\^{}(x-1) = 2\^{}(-3)

x - 1 = -3

\textbf{x = -2}

\paragraph{Exercício 5}
\label{sec:org4dd5e88}
Resolva a inequação: 5\^{}x < 125

\textbf{Solução:}

125 = 5³

5\^{}x < 5³

Base 5 > 1 (crescente) → mantém sinal

\textbf{x < 3}

\textbf{Resposta:} x < 3 ou x ∈ (-∞, 3)

\paragraph{Exercício 6}
\label{sec:orge89b8dc}
Simplifique: (x²y³)² / (xy)⁴

\textbf{Solução:}

= (x²)²(y³)² / x⁴y⁴ = x⁴y⁶ / x⁴y⁴ = x\^{}(4-4) y\^{}(6-4) = x⁰y² = y²

\textbf{Resposta:} y²

\subsubsection{Dicas para a Prova}
\label{sec:org02a729b}
\begin{enumerate}
\item \textbf{a⁰ = 1} (sempre)
\item \textbf{a\^{}(-n) = 1/a\^{}n} (expoente negativo = inverso)
\item \textbf{Multiplicação:} soma expoentes (mesma base)
\item \textbf{Divisão:} subtrai expoentes
\item \textbf{(a\^{}\{m)\}n = a\^{}(m×n)}
\item \textbf{Equação a\^{}x = a\^{}y:} x = y
\item \textbf{Inequação e a > 1:} mantém sinal
\item \textbf{Inequação e 0 < a < 1:} inverte sinal
\item \textbf{Gráfico:} sempre passa por (0, 1)
\end{enumerate}

\subsubsection{Conceitos-Chave para Memorizar}
\label{sec:orga89fcd5}
\textbf{Potenciação:} - a\^{}n = a × a × \ldots{} × a (n vezes) - a⁰ = 1 - a\^{}(-n) = 1/a\^{}n - a\^{}(m/n) = ⁿ√(a\^{}m)

\textbf{Propriedades (mesma base):} - a\^{}m × a\^{}n = a\^{}(m+n) - a\^{}m / a\^{}n = a\^{}(m-n) - (a\^{}\{m)\}n = a\^{}(m×n)

\textbf{Equação Exponencial:} - a\^{}x = a\^{}y → x = y

\textbf{Inequação:} - a > 1: crescente (mantém sinal) - 0 < a < 1: decrescente (inverte sinal)

\textbf{Gráfico:} - Passa por (0, 1) - Sempre f(x) > 0 - Eixo x é assíntota

\subsubsection{Fórmulas Essenciais}
\label{sec:orgfade194}
\begin{verbatim}
Propriedades de Potenciação:
a^m × a^n = a^(m+n)
a^m / a^n = a^(m-n)
(a^m)^n = a^(mn)
(ab)^n = a^n b^n
(a/b)^n = a^n / b^n

Casos Especiais:
a⁰ = 1
a¹ = a
a^(-n) = 1/a^n
a^(1/n) = ⁿ√a
a^(m/n) = ⁿ√(a^m)

Equação:
a^x = a^y  ⟹  x = y

Inequação:
a > 1:     a^x > a^y  ⟹  x > y
0 < a < 1: a^x > a^y  ⟹  x < y
\end{verbatim}

\subsubsection{Resumo Visual}
\label{sec:org5b63c6c}
\begin{verbatim}
PROPRIEDADES:

Multiplicação:    Divisão:
2³ × 2² = 2⁵     2⁵ / 2² = 2³
↑soma expoentes   ↑subtrai

Potência de Potência:
(2³)² = 2⁶
    ↑multiplica

GRÁFICOS:

a > 1 (crescente):   0 < a < 1 (decresc.):
     |                     |
   • |                   • |
  •  |(0,1)        (0,1)|  •
 •───•                 •───• 
     |                     |
─────┼─────           ─────┼─────
     
Sempre > 0           Sempre > 0
Passa por (0,1)      Passa por (0,1)
\end{verbatim}

\noindent\rule{\textwidth}{0.5pt}

\textbf{Tempo de estudo recomendado:} 90 minutos \textbf{Nível de dificuldade:} Médio \textbf{Importância para a prova:} ⭐⭐⭐⭐⭐ (essencial - função exponencial é muito cobrada!)

\subsection{Aula 35 - Química: Ligações Químicas - Parte 3 (Ligação Metálica) - 30min}
\label{sec:orge8f7061}
\subsubsection{Revisão: Ligações Químicas}
\label{sec:org4e5918f}
Já estudamos: - \textbf{Aula 31:} Ligação Iônica (metal + não-metal, transferência) - \textbf{Aula 33:} Ligação Covalente (não-metal + não-metal, compartilhamento)

\textbf{Nesta aula:} Ligação Metálica

\subsubsection{Ligação Metálica}
\label{sec:org48fc099}
\textbf{Definição:} Ligação entre \textbf{átomos de metais}, formando estruturas sólidas metálicas.

\textbf{Onde ocorre:} - Metais puros: Fe, Cu, Au, Ag, Al, Na, etc. - Ligas metálicas: bronze (Cu + Sn), aço (Fe + C), latão (Cu + Zn)

\subsubsection{Modelo do “Mar de Elétrons”}
\label{sec:org9a356fa}
\textbf{Teoria:} - Átomos metálicos perdem elétrons da camada de valência - Formam \textbf{cátions} (+) fixos em posições - Elétrons livres (“deslocalizados”) movem-se livremente - \textbf{“Mar de elétrons”:} elétrons circulam entre os cátions

\textbf{Representação:}

\begin{verbatim}
 ⁺   ⁺   ⁺   ⁺
e⁻ e⁻ e⁻ e⁻ e⁻
 ⁺   ⁺   ⁺   ⁺
e⁻ e⁻ e⁻ e⁻ e⁻
 ⁺   ⁺   ⁺   ⁺
\end{verbatim}

\textbf{Cátions (+):} núcleos dos átomos metálicos \textbf{e⁻:} elétrons livres, móveis

\textbf{Atração:} - Cátions (+) atraem elétrons (-) - Elétrons mantêm os cátions unidos - \textbf{Ligação metálica:} atração entre cátions e mar de elétrons

\subsubsection{Propriedades dos Metais}
\label{sec:org2e7d30e}
As propriedades dos metais são explicadas pelo modelo do mar de elétrons.

\paragraph{1. Condutividade Elétrica}
\label{sec:orgabdcf9f}
\textbf{Característica:} Metais \textbf{conduzem eletricidade} muito bem.

\textbf{Explicação:} - Elétrons livres movem-se facilmente - Ao aplicar diferença de potencial (voltagem), elétrons fluem - \textbf{Corrente elétrica} = fluxo de elétrons

\textbf{Aplicações:} Fios elétricos (Cu, Al)

\textbf{Melhores condutores:} Ag (prata), Cu (cobre), Au (ouro)

\paragraph{2. Condutividade Térmica}
\label{sec:org91ba0dc}
\textbf{Característica:} Metais conduzem \textbf{calor} eficientemente.

\textbf{Explicação:} - Elétrons livres transportam energia térmica rapidamente - Vibração dos cátions também contribui

\textbf{Aplicações:} Panelas (Al, Fe), dissipadores de calor (Cu)

\paragraph{3. Maleabilidade}
\label{sec:orgd30fdb3}
\textbf{Característica:} Metais podem ser \textbf{martelados} em lâminas finas sem quebrar.

\textbf{Explicação:} - Cátions podem deslizar uns sobre os outros - Mar de elétrons “se ajusta” à nova posição - Ligação não é rompida

\textbf{Exemplos:} Folhas de alumínio, ouro batido

\paragraph{4. Ductilidade}
\label{sec:orgef34b73}
\textbf{Característica:} Metais podem ser \textbf{esticados} em fios.

\textbf{Explicação:} - Mesma razão da maleabilidade - Cátions deslizam, elétrons mantêm ligação

\textbf{Exemplos:} Fios de cobre, arame

\paragraph{5. Brilho Metálico}
\label{sec:org0514fa2}
\textbf{Característica:} Metais têm \textbf{superfície brilhante} quando polidos.

\textbf{Explicação:} - Elétrons livres absorvem e re-emitem luz (reflexão) - Aparecem “brilhantes”

\textbf{Aplicações:} Espelhos (Ag), joias (Au, Ag)

\paragraph{6. Estado Físico e Pontos de Fusão/Ebulição}
\label{sec:org10cc418}
\textbf{Característica:} - Maioria são \textbf{sólidos} à temperatura ambiente (exceto Hg - mercúrio) - Pontos de fusão/ebulição \textbf{variados} (geralmente médios a altos)

\textbf{Explicação:} - Ligações metálicas são fortes (mas menos que iônicas) - Dependem do número de elétrons de valência e tamanho do átomo

\textbf{Exemplos:} - \textbf{Altos PF:} W (tungstênio) - 3422°C - \textbf{Baixos PF:} Hg (mercúrio) - líquido (-39°C), Ga (gálio) - 30°C

\paragraph{7. Densidade}
\label{sec:orgf9599fd}
\textbf{Característica:} Muitos metais são \textbf{densos}.

\textbf{Explicação:} - Átomos empacotados compactamente no retículo cristalino

\textbf{Exemplos:} - \textbf{Mais densos:} Os (ósmio), Ir (irídio), Pt (platina), Au (ouro) - \textbf{Menos densos:} Li (lítio), Na (sódio), K (potássio)

\subsubsection{Ligas Metálicas}
\label{sec:org0a18301}
\textbf{Liga metálica:} mistura de dois ou mais metais (ou metal + não-metal).

\textbf{Objetivo:} Melhorar propriedades (dureza, resistência à corrosão, etc.)

\textbf{Exemplos:}

\begin{center}
\begin{tabular}{lll}
Liga & Composição & Uso\\[0pt]
\hline
\textbf{Aço} & Fe + C (carbono) & Construção, ferramentas\\[0pt]
\textbf{Bronze} & Cu + Sn (estanho) & Esculturas, moedas antigas\\[0pt]
\textbf{Latão} & Cu + Zn (zinco) & Instrumentos musicais\\[0pt]
\textbf{Ouro 18k} & Au (75\%) + Cu/Ag (25\%) & Joias (ouro puro é mole)\\[0pt]
\textbf{Aço inox} & Fe + Cr + Ni & Talheres, equipamentos\\[0pt]
\end{tabular}
\end{center}

\textbf{Vantagens das ligas:} - \textbf{Mais duras} que metais puros - Resistência à \textbf{corrosão} - Podem ser \textbf{mais baratas} - Propriedades \textbf{ajustáveis}

\subsubsection{Resumo: Três Tipos de Ligação}
\label{sec:org149722a}
\begin{verbatim}
┌──────────┬─────────┬──────────┬──────────┐
│          │ IÔNICA  │COVALENTE │ METÁLICA │
├──────────┼─────────┼──────────┼──────────┤
│  Átomos  │Metal+   │Não-metal │  Metais  │
│          │Não-metal│+Não-metal│          │
├──────────┼─────────┼──────────┼──────────┤
│ Processo │Transfer.│Compartil.│Mar de e⁻ │
├──────────┼─────────┼──────────┼──────────┤
│  Formam  │  Íons   │Moléculas │  Rede    │
│          │cristais │          │ cristalina│
├──────────┼─────────┼──────────┼──────────┤
│   PF/PE  │  Altos  │  Baixos  │ Variados │
├──────────┼─────────┼──────────┼──────────┤
│Condutiv. │Fund/Dis.│   Não    │   Sim    │
│ Elétrica │         │          │ (sempre) │
├──────────┼─────────┼──────────┼──────────┤
│ Exemplo  │  NaCl   │   H₂O    │   Fe     │
└──────────┴─────────┴──────────┴──────────┘
\end{verbatim}

\subsubsection{Exercícios Resolvidos}
\label{sec:orga238721}
\paragraph{Exercício 1}
\label{sec:org2ad159a}
Por que metais conduzem eletricidade, mas compostos iônicos sólidos não?

\textbf{Resposta:} \textbf{Metais:} têm elétrons livres que se movem facilmente, transportando carga.

\textbf{Iônicos sólidos:} íons estão \textbf{fixos} no retículo cristalino, não podem se mover. Só conduzem quando fundidos ou dissolvidos (íons ficam livres).

\paragraph{Exercício 2}
\label{sec:orgc4b30c5}
Explique por que o ouro usado em joias é geralmente uma liga (ouro 18k) e não ouro puro.

\textbf{Resposta:} Ouro puro (24k) é muito \textbf{mole e maleável}, deformando facilmente. A liga com cobre ou prata (18k = 75\% Au + 25\% outros) torna o material \textbf{mais duro e resistente}, adequado para joias.

\paragraph{Exercício 3}
\label{sec:org5e26b52}
Qual o tipo de ligação presente em: a) Fe (ferro) b) H₂O (água) c) KCl (cloreto de potássio)

\textbf{Respostas:}

\begin{enumerate}
\item \textbf{Metálica} (Fe é metal)

\item \textbf{Covalente} (H e O são não-metais)

\item \textbf{Iônica} (K é metal, Cl é não-metal)
\end{enumerate}

\paragraph{Exercício 4}
\label{sec:org6aad363}
(UFMG) Por que o cobre é usado em fios elétricos?

\textbf{Resposta:} Cobre tem excelente \textbf{condutividade elétrica} (elétrons livres), é \textbf{maleável} e \textbf{ductil} (pode ser esticado em fios), e tem custo relativamente baixo comparado à prata (melhor condutor).

\subsubsection{Dicas para a Prova}
\label{sec:orgb4c7bf1}
\begin{enumerate}
\item \textbf{Ligação metálica:} entre metais
\item \textbf{Mar de elétrons:} elétrons livres, móveis
\item \textbf{Conduzem eletricidade:} sempre (elétrons livres)
\item \textbf{Maleáveis e dúcteis:} cátions deslizam
\item \textbf{Brilho metálico:} elétrons refletem luz
\item \textbf{Ligas:} mistura de metais (melhoram propriedades)
\item \textbf{Comparar ligações:} iônica, covalente, metálica
\item \textbf{Hg:} único metal líquido (temperatura ambiente)
\end{enumerate}

\subsubsection{Conceitos-Chave para Memorizar}
\label{sec:org1bb029a}
\textbf{Ligação Metálica:} - Entre metais - Mar de elétrons livres - Cátions fixos + elétrons móveis

\textbf{Propriedades dos Metais:} - Condutividade elétrica (elétrons livres) - Condutividade térmica - Maleabilidade (martelar) - Ductilidade (esticar) - Brilho metálico - Sólidos (maioria)

\textbf{Ligas Metálicas:} - Mistura de metais - Melhoram propriedades - Exemplos: aço, bronze, latão

\subsubsection{Resumo Visual}
\label{sec:org26ece8f}
\begin{verbatim}
LIGAÇÃO METÁLICA:

Metal  +  Metal  →  Retículo Metálico
  ↓        ↓
Perdem valência
  ↓
Cátions ⁺ + Mar de elétrons e⁻

Modelo:
   ⁺   ⁺   ⁺   ⁺
  e⁻ e⁻ e⁻ e⁻ e⁻  ← elétrons livres
   ⁺   ⁺   ⁺   ⁺
  e⁻ e⁻ e⁻ e⁻ e⁻
   ⁺   ⁺   ⁺   ⁺

Propriedades:
├─ Conduzem eletricidade (e⁻ livres)
├─ Conduzem calor
├─ Maleáveis (cátions deslizam)
├─ Dúcteis (esticam)
├─ Brilho (reflexão luz)
└─ Sólidos (maioria)

COMPARAÇÃO LIGAÇÕES:

IÔNICA:     Metal → Não-metal
            Transferência
            
COVALENTE:  Não-metal → Não-metal
            Compartilhamento
            
METÁLICA:   Metal → Metal
            Mar de elétrons
\end{verbatim}

\noindent\rule{\textwidth}{0.5pt}

\textbf{Tempo de estudo recomendado:} 30 minutos \textbf{Nível de dificuldade:} Médio \textbf{Importância para a prova:} ⭐⭐⭐⭐ (importante - completa o estudo de ligações!)

\section{11/29 - Férias Dia 4}
\label{sec:org337f379}
\subsection{Aula 36 - Matemática: Função Exponencial - Parte 2 (Equações, Gráficos e Crescimento) - 90min}
\label{sec:org8dfed90}
\subsubsection{Revisão: Função Exponencial Parte 1}
\label{sec:orgb5005aa}
Na Aula 34, estudamos: - Potenciação e propriedades - Equações exponenciais simples (bases iguais) - Inequações básicas - Gráfico introdutório

\textbf{Nesta aula:} Equações mais complexas, gráficos detalhados e aplicações.

\subsubsection{Definição de Função Exponencial}
\label{sec:orgdad71e2}
\textbf{Função Exponencial:}

\begin{verbatim}
f(x) = a^x
\end{verbatim}

Onde: - \textbf{a:} base (a > 0 e a ≠ 1) - \textbf{x:} expoente (variável)

\textbf{Por que a > 0 e a ≠ 1?} - \textbf{a > 0:} evitar resultados indefinidos (ex: (-2)\^{}(1/2)) - \textbf{a ≠ 1:} pois 1\^{}x = 1 (constante, não exponencial)

\textbf{Forma geral:}

\begin{verbatim}
f(x) = b · a^(cx + d) + e
\end{verbatim}

Mas começamos com a forma básica: f(x) = a\^{}x

\subsubsection{Propriedades da Função Exponencial}
\label{sec:org3721c19}
\textbf{1. Domínio:} D(f) = ℝ (todos os reais)

\textbf{2. Imagem:} Im(f) = ℝ₊* = (0, +∞) - Função sempre positiva: f(x) > 0 para todo x

\textbf{3. Intercepto com eixo y:} - Quando x = 0: f(0) = a⁰ = 1 - Sempre passa pelo ponto \textbf{(0, 1)}

\textbf{4. Não intercepta eixo x:} - Nunca f(x) = 0 (sempre positiva) - Eixo x é \textbf{assíntota horizontal}

\textbf{5. Injetora:} - Se a\^{}x₁ = a\^{}x₂, então x₁ = x₂ - Cada valor de y corresponde a único valor de x

\textbf{6. Monotonia (crescimento):} - \textbf{a > 1:} estritamente crescente - \textbf{0 < a < 1:} estritamente decrescente

\subsubsection{Gráficos Detalhados}
\label{sec:org9f62095}
\paragraph{Caso 1: a > 1 (Função Crescente)}
\label{sec:orgcaeed3d}
\textbf{Exemplos:} f(x) = 2\^{}x, f(x) = 3\^{}x, f(x) = 10\^{}x

\textbf{Características:} - Cresce rapidamente para x > 0 - Aproxima-se de 0 para x → -∞ - Passa por (0, 1)

\textbf{Tabela (f(x) = 2\^{}x):}

\begin{verbatim}
x  | -3  | -2  | -1  |  0  |  1  |  2  |  3  
f(x)| 1/8 | 1/4 | 1/2 |  1  |  2  |  4  |  8
\end{verbatim}

\textbf{Gráfico:}

\begin{verbatim}
    |
  8 |           •
  4 |         •
  2 |       •
  1 |•─────•─────────
1/2 |  •
1/4 | •
    |•________________
   -3  -1  0  1  2  3
\end{verbatim}

\textbf{Quanto maior a base, mais “vertical” a curva:} - 10\^{}x cresce mais rápido que 2\^{}x

\paragraph{Caso 2: 0 < a < 1 (Função Decrescente)}
\label{sec:org123d29b}
\textbf{Exemplos:} f(x) = (1/2)\^{}x, f(x) = (1/3)\^{}x

\textbf{Características:} - Decresce rapidamente para x > 0 - Aproxima-se de 0 para x → +∞ - Passa por (0, 1)

\textbf{Tabela (f(x) = (1/2)\^{}x):}

\begin{verbatim}
x  | -3  | -2  | -1  |  0  |  1  |  2  |  3
f(x)|  8  |  4  |  2  |  1  | 1/2 | 1/4 | 1/8
\end{verbatim}

\textbf{Gráfico:}

\begin{verbatim}
    |
  8 |•
  4 | •
  2 |  •
  1 |───•─────•
1/2 |       •  
1/4 |         •
    |___________•____
   -3  -1  0  1  2  3
\end{verbatim}

\textbf{Relação importante:}

\begin{verbatim}
(1/a)^x = a^(-x)
\end{verbatim}

Então: f(x) = (1/2)\^{}x é reflexão de g(x) = 2\^{}x em relação ao eixo y

\subsubsection{Equações Exponenciais Avançadas}
\label{sec:orgc90f653}
\paragraph{Técnica 1: Substituição}
\label{sec:orgd5befdd}
\textbf{Quando aparece a mesma base com expoentes relacionados.}

\textbf{Exemplo 1:} 4\^{}x - 2\^{}x - 2 = 0

\textbf{Observar:} 4\^{}x = (2²)\^{}x = 2\^{}(2x) = (2\^{}x)²

\textbf{Substituição:} y = 2\^{}x

y² - y - 2 = 0

(y - 2)(y + 1) = 0

y = 2 ou y = -1

\textbf{Voltar para x:}

2\^{}x = 2 → x = 1 ✓

2\^{}x = -1 → impossível (2\^{}x > 0) ✗

\textbf{Resposta:} x = 1

\textbf{Exemplo 2:} 9\^{}x - 4·3\^{}x + 3 = 0

9\^{}x = (3²)\^{}x = (3\^{}x)²

\textbf{Substituição:} y = 3\^{}x

y² - 4y + 3 = 0

(y - 1)(y - 3) = 0

y = 1 ou y = 3

3\^{}x = 1 = 3⁰ → x = 0 3\^{}x = 3 = 3¹ → x = 1

\textbf{Resposta:} x = 0 ou x = 1

\paragraph{Técnica 2: Bases Diferentes mas Relacionadas}
\label{sec:orgfac148d}
\textbf{Exemplo:} 2\^{}(x+1) + 2\^{}(x-1) = 5

\textbf{Fatorar usando propriedades:}

2\^{}(x+1) = 2·2\^{}x 2\^{}(x-1) = 2\^{}x/2

2·2\^{}x + 2\^{}x/2 = 5

Multiplicar por 2:

4·2\^{}x + 2\^{}x = 10

5·2\^{}x = 10

2\^{}x = 2 = 2¹

\textbf{x = 1}

\paragraph{Técnica 3: Logaritmo (introdução)}
\label{sec:org548b2ba}
Para equações como 2\^{}x = 5, precisaremos de logaritmos (próximas aulas).

\subsubsection{Inequações Exponenciais Avançadas}
\label{sec:org8a6849e}
\textbf{Exemplo 1:} 2\^{}(2x) - 5·2\^{}x + 4 ≤ 0

\textbf{Substituição:} y = 2\^{}x (y > 0)

y² - 5y + 4 ≤ 0

\textbf{Raízes:} (y - 1)(y - 4) = 0 y = 1 ou y = 4

\textbf{Estudo do sinal:} a = 1 > 0 (parábola ∪)

\begin{verbatim}
\____/
1    4
\end{verbatim}

y² - 5y + 4 ≤ 0: \textbf{1 ≤ y ≤ 4}

\textbf{Voltar para x:}

1 ≤ 2\^{}x ≤ 4

2⁰ ≤ 2\^{}x ≤ 2²

\textbf{0 ≤ x ≤ 2}

\textbf{Resposta:} S = [0, 2]

\subsubsection{Aplicações: Crescimento e Decaimento Exponencial}
\label{sec:org74b127c}
\paragraph{Aplicação 1: Crescimento Populacional}
\label{sec:orga1df25f}
Uma população de bactérias dobra a cada hora. Inicialmente há 100 bactérias.

\textbf{Modelo:}

\begin{verbatim}
P(t) = P₀ · 2^t
P(t) = 100 · 2^t
\end{verbatim}

Onde: - P(t): população após t horas - P₀ = 100: população inicial - Base 2: dobra a cada hora

\textbf{a) Quantas bactérias após 5 horas?}

P(5) = 100 · 2⁵ = 100 · 32 = 3.200 bactérias

\textbf{b) Quando atinge 12.800 bactérias?}

100 · 2\^{}t = 12.800

2\^{}t = 128 = 2⁷

t = 7 horas

\paragraph{Aplicação 2: Decaimento Radioativo}
\label{sec:org8504b15}
Uma substância radioativa tem meia-vida de 10 anos (metade se desintegra a cada 10 anos).

\textbf{Modelo:}

\begin{verbatim}
M(t) = M₀ · (1/2)^(t/10)
\end{verbatim}

Se M₀ = 800g:

\textbf{a) Massa após 30 anos?}

M(30) = 800 · (1/2)\^{}(30/10) = 800 · (1/2)³ = 800 · 1/8 = 100g

\textbf{b) Quando resta 100g?}

800 · (1/2)\^{}(t/10) = 100

(1/2)\^{}(t/10) = 1/8 = (1/2)³

t/10 = 3

t = 30 anos

\paragraph{Aplicação 3: Juros Compostos}
\label{sec:orge047c01}
Capital inicial de R\$ 1.000 aplicado a 10\% ao ano (juros compostos).

\textbf{Modelo:}

\begin{verbatim}
M(t) = C · (1 + i)^t
M(t) = 1000 · (1,1)^t
\end{verbatim}

\textbf{Após 5 anos:}

M(5) = 1000 · (1,1)⁵ ≈ 1000 · 1,61 ≈ R\$ 1.610

\subsubsection{Exercícios Resolvidos}
\label{sec:org1557783}
\paragraph{Exercício 1}
\label{sec:org5ab5d1f}
Resolva: 5\^{}(2x) - 25 = 0

\textbf{Solução:}

5\^{}(2x) = 25 = 5²

2x = 2

\textbf{x = 1}

\paragraph{Exercício 2}
\label{sec:org4deb3b5}
Determine o conjunto solução: 3\^{}x > 27

\textbf{Solução:}

3\^{}x > 3³

Base 3 > 1 (crescente) → mantém sinal

\textbf{x > 3}

\textbf{S = (3, +∞)}

\paragraph{Exercício 3}
\label{sec:orgf7f90d7}
Resolva: 2\^{}(x+2) + 2\^{}x = 20

\textbf{Solução:}

2\^{}(x+2) = 2² · 2\^{}x = 4 · 2\^{}x

4·2\^{}x + 2\^{}x = 20

5·2\^{}x = 20

2\^{}x = 4 = 2²

\textbf{x = 2}

\paragraph{Exercício 4}
\label{sec:orgbe7d8f7}
Uma população de 500 bactérias triplica a cada 2 horas. Quantas bactérias após 6 horas?

\textbf{Solução:}

P(t) = 500 · 3\^{}(t/2)

P(6) = 500 · 3\^{}(6/2) = 500 · 3³ = 500 · 27 = 13.500

\textbf{Resposta:} 13.500 bactérias

\subsubsection{Dicas para a Prova}
\label{sec:org057e88b}
\begin{enumerate}
\item \textbf{Sempre passa por (0, 1)}
\item \textbf{Sempre positiva:} f(x) > 0
\item \textbf{a > 1:} crescente; \textbf{0 < a < 1:} decrescente
\item \textbf{Substituição:} quando aparece (a\^{}x)²
\item \textbf{Fatorar:} 2\^{}(x+1) = 2 · 2\^{}x
\item \textbf{Crescimento:} base > 1
\item \textbf{Decaimento:} base entre 0 e 1
\item \textbf{Juros compostos:} M = C(1+i)\^{}t
\end{enumerate}

\subsubsection{Conceitos-Chave para Memorizar}
\label{sec:org68d1789}
\textbf{Função Exponencial:} - f(x) = a\^{}x (a > 0, a ≠ 1) - Domínio: ℝ - Imagem: (0, +∞) - Passa por (0, 1)

\textbf{Comportamento:} - a > 1: crescente - 0 < a < 1: decrescente

\textbf{Técnicas de Resolução:} - Bases iguais: igualar expoentes - Substituição: y = a\^{}x - Fatorar: usar propriedades

\textbf{Aplicações:} - Crescimento: P(t) = P₀ · a\^{}t (a > 1) - Decaimento: M(t) = M₀ · a\^{}t (0 < a < 1) - Juros: M = C(1+i)\^{}t

\subsubsection{Fórmulas Essenciais}
\label{sec:orgc40c420}
\begin{verbatim}
Função Exponencial:
f(x) = a^x  (a > 0, a ≠ 1)

Propriedades:
- D(f) = ℝ
- Im(f) = ℝ₊* = (0, +∞)
- f(0) = 1
- a > 1: crescente
- 0 < a < 1: decrescente

Transformações:
a^(x+n) = a^n · a^x
a^(x-n) = a^x / a^n
(a^m)^x = a^(mx)

Aplicações:
Crescimento: P(t) = P₀ · a^t
Decaimento: M(t) = M₀ · (1/2)^(t/T)  [T = meia-vida]
Juros Compostos: M = C(1 + i)^t
\end{verbatim}

\subsubsection{Resumo Visual}
\label{sec:orgafc5c56}
\begin{verbatim}
GRÁFICOS:

a > 1 (crescente):     0 < a < 1 (decresc.):
      |                      |
    8 |        •           8 |•
    4 |      •             4 | •
    2 |    •               2 |  •
    1 |──•──               1 |───•──
      |•                     |       •
    ──┼────────           ──┼────────
      
CRESCIMENTO:           DECAIMENTO:
P(t) = P₀ · a^t       M(t) = M₀ · a^t
(a > 1)               (0 < a < 1)

  •                      •
    •                  •
      •              •
        •          •
          •      •

SUBSTITUIÇÃO:
4^x = (2^x)²
9^x = (3^x)²
Se y = a^x, então a^(2x) = y²
\end{verbatim}

\noindent\rule{\textwidth}{0.5pt}

\textbf{Tempo de estudo recomendado:} 90 minutos \textbf{Nível de dificuldade:} Médio-Alto \textbf{Importância para a prova:} ⭐⭐⭐⭐⭐ (essencial - aplicações são muito cobradas!)

\subsection{Aula 37 - Física: Forças Especiais - Peso, Normal e Atrito (Revisão e Aprofundamento) - 30min}
\label{sec:org8f64e4a}
\subsubsection{Revisão: Leis de Newton}
\label{sec:org8b7dc5b}
Na Aula 22, estudamos as Leis de Newton e tipos básicos de forças.

\textbf{Nesta aula:} Aprofundamento em forças especiais e aplicações.

\subsubsection{Força Peso (P)}
\label{sec:org788ee6d}
\textbf{Definição:} Força gravitacional que a Terra exerce sobre um corpo.

\textbf{Fórmula:}

\begin{verbatim}
P = m · g
\end{verbatim}

Onde: - P: peso (N - newtons) - m: massa (kg) - g: aceleração da gravidade (m/s²)

\textbf{Na Terra:} g ≈ 10 m/s² (ou 9,8 m/s² para maior precisão)

\textbf{Características:} - \textbf{Direção:} vertical - \textbf{Sentido:} para baixo (centro da Terra) - \textbf{Ponto de aplicação:} centro de gravidade do corpo

\textbf{Diferença Massa vs. Peso:}

\begin{center}
\begin{tabular}{ll}
Massa (m) & Peso (P)\\[0pt]
\hline
Quantidade de matéria & Força gravitacional\\[0pt]
Escalar (kg) & Vetor (N)\\[0pt]
NÃO varia & VARIA com g\\[0pt]
Mesma em qualquer lugar & Diferente em planetas\\[0pt]
\end{tabular}
\end{center}

\textbf{Exemplos:}

Pessoa com 60 kg: - \textbf{Terra:} P = 60 × 10 = 600 N - \textbf{Lua:} g\_Lua ≈ 1,6 m/s² → P = 60 × 1,6 = 96 N - \textbf{Massa:} 60 kg (não muda!)

\subsubsection{Força Normal (N)}
\label{sec:orgc0c5478}
\textbf{Definição:} Força de reação da superfície, perpendicular ao contato.

\textbf{Características:} - \textbf{Direção:} perpendicular à superfície - \textbf{Sentido:} “empurra” o corpo para fora da superfície - \textbf{Intensidade:} varia conforme situação

\textbf{NÃO é sempre igual ao peso!}

\paragraph{Caso 1: Superfície Horizontal (equilíbrio vertical)}
\label{sec:org6a9da31}
\begin{verbatim}
  ↑ N
  |
[corpo]
  |
  ↓ P
\end{verbatim}

\textbf{Equilíbrio vertical:} ΣF\_y = 0 N - P = 0 \textbf{N = P = mg}

\paragraph{Caso 2: Corpo em Aceleração Vertical}
\label{sec:org34ad52a}
\textbf{a) Elevador subindo acelerado (a ↑):}

\begin{verbatim}
  ↑ N
  |
[corpo] ↑ a
  |
  ↓ P
\end{verbatim}

F\_R = ma (para cima) N - P = ma \textbf{N = P + ma = m(g + a)}

Normal \textbf{maior} que peso (sensação de “mais pesado”)

\textbf{b) Elevador descendo acelerado (a ↓):}

F\_R = ma (para baixo) P - N = ma \textbf{N = P - ma = m(g - a)}

Normal \textbf{menor} que peso (sensação de “mais leve”)

\textbf{c) Queda livre (a = g ↓):}

N = m(g - g) = 0

\textbf{N = 0} (ausência de peso, “flutuação”)

\paragraph{Caso 3: Plano Inclinado}
\label{sec:org3f7f827}
\begin{verbatim}
        N ↑ (perp. ao plano)
       /|
      / |
   [corpo]
    / | \
   /  |  \ P_y (paralela)
  /   |   \
 /    ↓    \
/   P_x     \ θ
──────────────
\end{verbatim}

\textbf{Decomposição do peso:} - P\_x = P · sen(θ) = mg sen(θ) (paralelo ao plano) - P\_y = P · cos(θ) = mg cos(θ) (perpendicular ao plano)

\textbf{Equilíbrio perpendicular:} \textbf{N = P\_y = mg cos(θ)}

\paragraph{Caso 4: Corpo Pressionado Contra Parede}
\label{sec:org7afc7e4}
\begin{verbatim}
Parede
  │ ← N
  │
  │ [corpo] → F (força aplicada)
  │
\end{verbatim}

\textbf{N = F} (força aplicada horizontalmente)

\subsubsection{Força de Atrito (F\_at)}
\label{sec:orgd8a32a1}
\textbf{Definição:} Força que se opõe ao movimento relativo entre superfícies em contato.

\textbf{Características:} - \textbf{Direção:} paralela às superfícies - \textbf{Sentido:} oposto ao movimento (ou tendência) - \textbf{Origem:} irregularidades microscópicas das superfícies

\paragraph{Atrito Estático (F\_at,e)}
\label{sec:orgcec3f6a}
\textbf{Quando:} corpo em repouso, tende a se mover mas ainda está parado.

\textbf{Características:} - Varia de 0 até máximo - \textbf{F\_at,e ≤ μ\_e · N}

\textbf{Força de atrito estático máximo:}

\begin{verbatim}
F_at,e(máx) = μ_e · N
\end{verbatim}

\textbf{μ\_e:} coeficiente de atrito estático (adimensional)

\textbf{Exemplo:} Empurrando caixa com força crescente: - Força pequena: caixa não move (F\_at = F\_aplicada) - Aumenta força: atrito aumenta - Atinge F\_at(máx): caixa está prestes a deslizar - Ultrapassa F\_at(máx): caixa começa a deslizar

\paragraph{Atrito Cinético (F\_at,c)}
\label{sec:orgef1fc6a}
\textbf{Quando:} corpo em movimento.

\textbf{Fórmula:}

\begin{verbatim}
F_at,c = μ_c · N
\end{verbatim}

\textbf{μ\_c:} coeficiente de atrito cinético

\textbf{Observações:} - Valor constante (não varia) - Geralmente μ\_c < μ\_e (mais fácil manter movimento que iniciar)

\paragraph{Fatores que Influenciam o Atrito}
\label{sec:orgefe7805}
\textbf{Depende de:} - \textbf{Natureza das superfícies:} μ (coeficiente) - \textbf{Força normal:} N

\textbf{NÃO depende de:} - Área de contato - Velocidade (atrito cinético)

\subsubsection{Aplicações}
\label{sec:org9184715}
\paragraph{Problema 1}
\label{sec:org560df34}
Um bloco de 10 kg está sobre uma superfície horizontal. μ\_e = 0,4 e μ\_c = 0,3. Qual a força mínima para: a) Tirar o bloco do repouso? b) Mantê-lo em movimento? (g = 10 m/s²)

\textbf{Solução:}

N = P = mg = 10 × 10 = 100 N

\begin{enumerate}
\item \textbf{Força para tirar do repouso:} F ≥ F\_at,e(máx) = μ\_e · N = 0,4 × 100 = 40 N
\end{enumerate}

\textbf{F ≥ 40 N}

\begin{enumerate}
\setcounter{enumi}{1}
\item \textbf{Manter em movimento:} F = F\_at,c = μ\_c · N = 0,3 × 100 = 30 N
\end{enumerate}

\textbf{F = 30 N}

\paragraph{Problema 2}
\label{sec:org2eda443}
Bloco de 5 kg em plano inclinado de 30°. Determine a normal e a componente paralela do peso. (g = 10 m/s²)

\textbf{Solução:}

P = 5 × 10 = 50 N

\textbf{Normal:} N = mg cos(30°) = 50 × (√3/2) ≈ 50 × 0,87 ≈ 43,5 N

\textbf{Componente paralela:} P\_x = mg sen(30°) = 50 × 0,5 = 25 N

\subsubsection{Exercícios Resolvidos}
\label{sec:org70e306f}
\paragraph{Exercício 1}
\label{sec:org23feaf8}
Qual o peso de um corpo de massa 8 kg na Terra (g = 10 m/s²)?

\textbf{Solução:} P = mg = 8 × 10 = 80 N

\textbf{Resposta:} 80 N

\paragraph{Exercício 2}
\label{sec:org1b475e9}
Uma pessoa de 70 kg está em um elevador que sobe com aceleração de 2 m/s². Qual a normal? (g = 10 m/s²)

\textbf{Solução:}

Elevador subindo acelerado: N = m(g + a) = 70(10 + 2) = 70 × 12 = 840 N

\textbf{Resposta:} 840 N

\paragraph{Exercício 3}
\label{sec:org00e17d1}
(UFMG) Um bloco está em repouso sobre uma mesa. A força normal é: a) Sempre igual ao peso b) Par ação-reação do peso c) Força de reação da mesa

\textbf{Resposta:} \textbf{c) Força de reação da mesa} (perpendicular à superfície)

\textbf{Importante:} Normal NÃO é par ação-reação do peso! - Peso: Terra atrai bloco - Reação do peso: Bloco atrai Terra

\subsubsection{Dicas para a Prova}
\label{sec:org174585f}
\begin{enumerate}
\item \textbf{Peso:} sempre P = mg (para baixo)
\item \textbf{Normal:} perpendicular à superfície
\item \textbf{N = P:} só em superfície horizontal sem aceleração vertical
\item \textbf{Plano inclinado:} N = mg cos(θ)
\item \textbf{Atrito estático:} F\_at ≤ μ\_e N (varia)
\item \textbf{Atrito cinético:} F\_at = μ\_c N (constante)
\item \textbf{μ\_e > μ\_c:} iniciar movimento é mais difícil
\item \textbf{Atrito:} sempre oposto ao movimento
\end{enumerate}

\subsubsection{Conceitos-Chave para Memorizar}
\label{sec:orgb27dd33}
\textbf{Peso:} - P = mg - Vertical, para baixo - Varia com g

\textbf{Normal:} - Perpendicular à superfície - Varia conforme situação - Horizontal: N = P - Inclinado: N = mg cos(θ)

\textbf{Atrito:} - Opõe movimento - Estático: F\_at ≤ μ\_e N - Cinético: F\_at = μ\_c N - μ\_e > μ\_c

\subsubsection{Fórmulas Essenciais}
\label{sec:org1dbc492}
\begin{verbatim}
Peso:
P = m · g

Normal (casos):
Horizontal: N = P = mg
Elevador (↑a): N = m(g + a)
Elevador (↓a): N = m(g - a)
Plano inclinado: N = mg cos(θ)

Atrito:
Estático (máx): F_at,e = μ_e · N
Cinético: F_at,c = μ_c · N

Plano Inclinado:
P_paralela = mg sen(θ)
P_perpendicular = mg cos(θ)
\end{verbatim}

\subsubsection{Resumo Visual}
\label{sec:orgb2f8fef}
\begin{verbatim}
PESO:
    [corpo]
       |
       ↓ P = mg
       
NORMAL:

Horizontal:    Inclinado (θ):
  ↑ N             N ↑
  |              /|
[corpo]      [corpo]
  |            /
  ↓ P         / P
           θ
           
ATRITO:

Estático:           Cinético:
[corpo] → F_apl   [corpo] → v
←F_at (varia)     ←F_at (constante)

F_at ≤ μ_e·N      F_at = μ_c·N
\end{verbatim}

\noindent\rule{\textwidth}{0.5pt}

\textbf{Tempo de estudo recomendado:} 30 minutos \textbf{Nível de dificuldade:} Médio \textbf{Importância para a prova:} ⭐⭐⭐⭐ (muito importante - forças são essenciais!)

\section{11/30 - Férias Dia 5}
\label{sec:org26e4bef}
\subsection{Aula 38 - Matemática: Função Logarítmica - Parte 1 (Definição e Logaritmo Decimal/Natural) - 90min}
\label{sec:org442d311}
\subsubsection{Introdução: O Problema Inverso}
\label{sec:orgd156dd9}
\textbf{Problema:} Resolver 2\^{}x = 8 é fácil (x = 3).

\textbf{Mas e 2\^{}x = 5?} Não conseguimos escrever 5 como potência de 2!

\textbf{Solução:} Usar \textbf{logaritmos}.

\subsubsection{Definição de Logaritmo}
\label{sec:org5affd30}
\textbf{Logaritmo} é o expoente ao qual devemos elevar a base para obter um número.

\textbf{Definição formal:}

\begin{verbatim}
log_a b = x  ⟺  a^x = b
\end{verbatim}

Onde: - \textbf{a:} base (a > 0, a ≠ 1) - \textbf{b:} logaritmando (b > 0) - \textbf{x:} logaritmo

\textbf{Lê-se:} “Logaritmo de b na base a é igual a x”

\textbf{Significado:} “a elevado a x é igual a b”

\subsubsection{Exemplos Básicos}
\label{sec:org762da4e}
\textbf{Exemplo 1:} log₂ 8 = ?

“2 elevado a quanto dá 8?” 2³ = 8 \textbf{log₂ 8 = 3}

\textbf{Exemplo 2:} log₅ 125 = ?

5³ = 125 \textbf{log₅ 125 = 3}

\textbf{Exemplo 3:} log₁₀ 100 = ?

10² = 100 \textbf{log₁₀ 100 = 2}

\textbf{Exemplo 4:} log₃ (1/9) = ?

3\^{}x = 1/9 = 1/3² = 3⁻² \textbf{log₃ (1/9) = -2}

\textbf{Exemplo 5:} log₄ 2 = ?

4\^{}x = 2 (2²)\^{}x = 2 2\^{}(2x) = 2¹ 2x = 1 \textbf{log₄ 2 = 1/2}

\subsubsection{Casos Especiais}
\label{sec:org4505113}
\paragraph{1. Logaritmo de 1}
\label{sec:org4874093}
\begin{verbatim}
log_a 1 = 0  (para qualquer a)
\end{verbatim}

\textbf{Por quê?} a⁰ = 1

\textbf{Exemplos:} - log₅ 1 = 0 - log₁₀ 1 = 0

\paragraph{2. Logaritmo da Base}
\label{sec:org11d596e}
\begin{verbatim}
log_a a = 1  (para qualquer a)
\end{verbatim}

\textbf{Por quê?} a¹ = a

\textbf{Exemplos:} - log₇ 7 = 1 - log₁₀ 10 = 1

\paragraph{3. Potência da Base}
\label{sec:org90ab39f}
\begin{verbatim}
log_a (a^n) = n
\end{verbatim}

\textbf{Exemplos:} - log₂ (2⁵) = 5 - log₃ (3⁻²) = -2

\paragraph{4. Base Elevada ao Logaritmo}
\label{sec:org207232c}
\begin{verbatim}
a^(log_a b) = b
\end{verbatim}

\textbf{Exemplo:} 2\^{}(log₂ 5) = 5

\subsubsection{Condições de Existência}
\label{sec:org4ab57d5}
\textbf{Para log\_a b existir:}

\begin{enumerate}
\item \textbf{a > 0 e a ≠ 1} (base positiva e diferente de 1)
\item \textbf{b > 0} (logaritmando positivo)
\end{enumerate}

\textbf{Consequências:} - log₂ (-4): NÃO existe (logaritmando negativo) - log₁ 5: NÃO existe (base = 1) - log₋₂ 8: NÃO existe (base negativa) - log₂ 0: NÃO existe (logaritmando = 0)

\subsubsection{Logaritmo Decimal (Base 10)}
\label{sec:org4b8ef97}
\textbf{Logaritmo decimal:} log₁₀ b

\textbf{Notação simplificada:}

\begin{verbatim}
log b = log₁₀ b
\end{verbatim}

Quando a base não aparece, é base 10!

\textbf{Exemplos:} - log 100 = log₁₀ 100 = 2 (pois 10² = 100) - log 1000 = 3 (pois 10³ = 1000) - log 0,01 = -2 (pois 10⁻² = 0,01)

\textbf{Aplicações:} - Escala Richter (terremotos) - pH (acidez) - Decibéis (som) - Ordens de grandeza

\subsubsection{Logaritmo Natural (Base e)}
\label{sec:org3876f59}
\textbf{Número de Euler:}

\begin{verbatim}
e ≈ 2,718281828...
\end{verbatim}

\textbf{Logaritmo natural:} log\_e b

\textbf{Notação:}

\begin{verbatim}
ln b = log_e b
\end{verbatim}

\textbf{Lê-se:} “ene de b” ou “logaritmo natural de b”

\textbf{Exemplos:} - ln e = 1 (pois e¹ = e) - ln e² = 2 - ln 1 = 0

\textbf{Aplicações:} - Crescimento populacional - Juros compostos contínuos - Decaimento radioativo - Cálculo (derivadas, integrais)

\textbf{Por que “natural”?} Aparece naturalmente em problemas de crescimento/decaimento contínuo.

\subsubsection{Relação entre Logaritmo e Exponencial}
\label{sec:org05b9a9d}
\textbf{Funções inversas:}

Se f(x) = a\^{}x (exponencial) Então f⁻¹(x) = log\_a x (logarítmica)

\textbf{Propriedade fundamental:}

\begin{verbatim}
y = a^x  ⟺  x = log_a y
\end{verbatim}

\textbf{Consequências:} - log\_a (a\^{}x) = x - a\^{}(log\_a x) = x

\subsubsection{Gráfico da Função Logarítmica (Introdução)}
\label{sec:orgae2e517}
\textbf{Função:} f(x) = log\_a x

\paragraph{Caso 1: a > 1}
\label{sec:org9f15cfa}
\textbf{Exemplo:} f(x) = log₂ x

\begin{verbatim}
x    | 1/4 | 1/2 | 1  | 2  | 4  | 8
log₂x| -2  | -1  | 0  | 1  | 2  | 3
\end{verbatim}

\textbf{Gráfico:}

\begin{verbatim}
   |
 3 |           •
 2 |       •
 1 |   • 
 0 |─•─────────
-1 | •
-2 |•
   |____________
     1  2  4  8
\end{verbatim}

\textbf{Características:} - \textbf{Crescente} - Passa por (1, 0) - Eixo y é assíntota vertical - Domínio: x > 0 - Imagem: ℝ

\paragraph{Caso 2: 0 < a < 1}
\label{sec:orgced881d}
\textbf{Exemplo:} f(x) = log\_(1/2) x

\textbf{Características:} - \textbf{Decrescente} - Passa por (1, 0) - Eixo y é assíntota

\subsubsection{Exercícios Resolvidos}
\label{sec:org06f5616}
\paragraph{Exercício 1}
\label{sec:org0ed7d49}
Calcule: a) log₅ 25 b) log₃ 27 c) log₁₀ 0,001

\textbf{Soluções:}

\begin{enumerate}
\item 5\^{}x = 25 = 5² → \textbf{x = 2}

\item 3\^{}x = 27 = 3³ → \textbf{x = 3}

\item 10\^{}x = 0,001 = 1/1000 = 10⁻³ → \textbf{x = -3}
\end{enumerate}

\paragraph{Exercício 2}
\label{sec:org07cfd90}
Determine o valor de log₄ 8.

\textbf{Solução:}

4\^{}x = 8 (2²)\^{}x = 2³ 2\^{}(2x) = 2³ 2x = 3 \textbf{x = 3/2}

\paragraph{Exercício 3}
\label{sec:orge578e25}
Calcule: log₇ 1 + log₇ 7 + log₇ 49

\textbf{Solução:}

log₇ 1 = 0 log₇ 7 = 1 log₇ 49 = log₇ (7²) = 2

Soma: 0 + 1 + 2 = \textbf{3}

\paragraph{Exercício 4}
\label{sec:orgca3061d}
Para quais valores de x existe log₂ (x - 3)?

\textbf{Solução:}

\textbf{Condição:} logaritmando > 0

x - 3 > 0 \textbf{x > 3}

\paragraph{Exercício 5}
\label{sec:org95423e1}
Simplifique: 3\^{}(log₃ 10)

\textbf{Solução:}

Propriedade: a\^{}(log\_a b) = b

\textbf{3\^{}(log₃ 10) = 10}

\subsubsection{Dicas para a Prova}
\label{sec:orgbd176b7}
\begin{enumerate}
\item \textbf{log\_a b = x ⟺ a\^{}x = b} (definição!)
\item \textbf{log\_a 1 = 0} (sempre)
\item \textbf{log\_a a = 1} (sempre)
\item \textbf{log\_a (a\^{}n) = n}
\item \textbf{a\^{}(log\_a b) = b}
\item \textbf{log b = log₁₀ b} (base 10)
\item \textbf{ln b = log\_e b} (base e)
\item \textbf{Condições:} a > 0, a ≠ 1, b > 0
\end{enumerate}

\subsubsection{Conceitos-Chave para Memorizar}
\label{sec:orgbc7e637}
\textbf{Definição:} - log\_a b = x ⟺ a\^{}x = b - a: base - b: logaritmando - x: logaritmo

\textbf{Casos Especiais:} - log\_a 1 = 0 - log\_a a = 1 - log\_a (a\^{}n) = n - a\^{}(log\_a b) = b

\textbf{Bases Especiais:} - log b = log₁₀ b (decimal) - ln b = log\_e b (natural)

\textbf{Condições:} - a > 0, a ≠ 1 - b > 0

\subsubsection{Fórmulas Essenciais}
\label{sec:orgc6f0b39}
\begin{verbatim}
Definição:
log_a b = x  ⟺  a^x = b

Casos Especiais:
log_a 1 = 0
log_a a = 1
log_a (a^n) = n
a^(log_a b) = b

Logaritmos Especiais:
log b = log₁₀ b  (decimal)
ln b = log_e b   (natural, e ≈ 2,718)

Condições de Existência:
a > 0, a ≠ 1
b > 0
\end{verbatim}

\subsubsection{Resumo Visual}
\label{sec:org6d7e97e}
\begin{verbatim}
DEFINIÇÃO:

log_a b = x
     ↓
   a^x = b
   
"a elevado a x dá b"

EXEMPLOS:
log₂ 8 = 3  porque  2³ = 8
log₁₀ 100 = 2  porque  10² = 100

GRÁFICO (a > 1):
    |
  2 |       •
  1 |   •
  0 |─•──────
 -1 | •
    |____________
      1  2  4

- Crescente (a > 1)
- Passa por (1,0)
- Assíntota: x = 0
\end{verbatim}

\noindent\rule{\textwidth}{0.5pt}

\textbf{Tempo de estudo recomendado:} 90 minutos \textbf{Nível de dificuldade:} Médio \textbf{Importância para a prova:} ⭐⭐⭐⭐⭐ (essencial - logaritmos são fundamentais!)

\subsection{Aula 39 - Física: Plano Inclinado - 30min}
\label{sec:org8e9567d}
\subsubsection{Introdução}
\label{sec:org8e40926}
\textbf{Plano inclinado:} superfície plana formando ângulo θ com a horizontal.

\textbf{Exemplos:} rampas, ladeiras, escorregadores

\textbf{Importância:} Combina conceitos de forças, decomposição vetorial, atrito.

\subsubsection{Forças no Plano Inclinado}
\label{sec:orga032f02}
\begin{verbatim}
        N ↑ (normal)
       /|
      / |
   [m]  |
    / | \
   /  |  \ F_at (atrito, se houver)
  /   |   \
 /    ↓    \
/     P     \ θ
──────────────
\end{verbatim}

\textbf{Forças atuantes:} 1. \textbf{Peso (P):} vertical, para baixo (P = mg) 2. \textbf{Normal (N):} perpendicular ao plano 3. \textbf{Atrito (F\_at):} paralelo ao plano, opõe movimento

\subsubsection{Decomposição do Peso}
\label{sec:org5032953}
Como o peso é vertical mas o plano é inclinado, decompomos P:

\textbf{Componentes do peso:}

\begin{verbatim}
P_x = P sen(θ) = mg sen(θ)  (paralela ao plano, "desce")
P_y = P cos(θ) = mg cos(θ)  (perpendicular ao plano)
\end{verbatim}

\textbf{Diagrama:}

\begin{verbatim}
  P_y
   ↑
   |
   |___P_x →
  /θ
 /
/ P
\end{verbatim}

\textbf{Memorização:} - \textbf{P\_x (paralela):} sen(θ) - “sobe” com θ - \textbf{P\_y (perpendicular):} cos(θ) - “desce” com θ

\subsubsection{Força Normal no Plano Inclinado}
\label{sec:org5f05e68}
\textbf{Equilíbrio perpendicular ao plano:}

Não há movimento perpendicular → ΣF\_perp = 0

\textbf{N - P\_y = 0}

\begin{verbatim}
N = P_y = P cos(θ) = mg cos(θ)
\end{verbatim}

\textbf{Importante:} N ≠ P (diferente da superfície horizontal!)

\textbf{Observação:} Quanto maior θ, menor N.

\subsubsection{Casos de Movimento}
\label{sec:org59f57ee}
\paragraph{Caso 1: Plano Inclinado SEM Atrito}
\label{sec:org7a5c040}
\textbf{Forças paralelas ao plano:} - P\_x (desce)

\textbf{2ª Lei de Newton (paralelo ao plano):}

\begin{verbatim}
F_R = ma
P_x = ma
mg sen(θ) = ma
a = g sen(θ)
\end{verbatim}

\textbf{Aceleração:}

\begin{verbatim}
a = g sen(θ)  (descendo o plano)
\end{verbatim}

\textbf{Observações:} - Aceleração independe da massa! - θ = 0°: a = 0 (horizontal) - θ = 90°: a = g (queda livre)

\paragraph{Caso 2: Plano Inclinado COM Atrito}
\label{sec:orgb7f8fe1}
\textbf{Forças paralelas:} - P\_x (desce) - F\_at (sobe, opõe movimento)

\textbf{Atrito:} F\_at = μN = μ mg cos(θ)

\textbf{2ª Lei (descendo):}

\begin{verbatim}
mg sen(θ) - μ mg cos(θ) = ma
a = g(sen(θ) - μ cos(θ))
\end{verbatim}

\textbf{Condição para descer:} sen(θ) > μ cos(θ) \textbf{tan(θ) > μ}

Se tan(θ) ≤ μ: corpo não desce (atrito segura)

\paragraph{Caso 3: Corpo Empurrado Plano Acima}
\label{sec:org990010b}
\textbf{Força aplicada F (paralela ao plano, subindo):}

\textbf{Subindo:}

\begin{verbatim}
F - P_x - F_at = ma
F - mg sen(θ) - μ mg cos(θ) = ma
\end{verbatim}

\subsubsection{Aplicações}
\label{sec:org82c9782}
\paragraph{Problema 1}
\label{sec:org7476ece}
Bloco de 10 kg em plano inclinado de 30°, sem atrito. Qual a aceleração? (g = 10 m/s²)

\textbf{Solução:}

a = g sen(30°) = 10 × 0,5 = 5 m/s²

\textbf{Resposta:} 5 m/s² (descendo o plano)

\paragraph{Problema 2}
\label{sec:org1df55c4}
Mesmo bloco, mesma inclinação, μ = 0,2. Qual a aceleração?

\textbf{Solução:}

a = g(sen(θ) - μ cos(θ)) a = 10(sen(30°) - 0,2 × cos(30°)) a = 10(0,5 - 0,2 × 0,87) a = 10(0,5 - 0,174) a = 10 × 0,326 a ≈ 3,26 m/s²

\textbf{Resposta:} ≈ 3,3 m/s²

\paragraph{Problema 3}
\label{sec:org2e723fd}
Determine a normal de um bloco de 5 kg em plano de 60°. (g = 10 m/s²)

\textbf{Solução:}

N = mg cos(60°) = 5 × 10 × 0,5 = 25 N

\textbf{Resposta:} 25 N

\subsubsection{Exercícios Resolvidos}
\label{sec:org5396f0f}
\paragraph{Exercício 1}
\label{sec:orgf999051}
Em um plano inclinado de 45° sem atrito, um corpo de 2 kg tem qual aceleração? (g = 10 m/s²)

\textbf{Solução:}

a = g sen(45°) = 10 × (√2/2) ≈ 10 × 0,7 = 7 m/s²

\textbf{Resposta:} ≈ 7 m/s²

\paragraph{Exercício 2}
\label{sec:org11ba719}
(UFMG) Qual a força normal em um corpo de 8 kg em plano de 30°?

\textbf{Solução:}

N = mg cos(30°) = 8 × 10 × (√3/2) ≈ 80 × 0,87 ≈ 69,6 N

\textbf{Resposta:} ≈ 70 N

\subsubsection{Dicas para a Prova}
\label{sec:org2a64d35}
\begin{enumerate}
\item \textbf{Decompor peso:} P\_x = mg sen(θ), P\_y = mg cos(θ)
\item \textbf{Normal:} N = mg cos(θ) (não é igual a P!)
\item \textbf{Sem atrito:} a = g sen(θ)
\item \textbf{Com atrito:} a = g(sen(θ) - μ cos(θ))
\item \textbf{Ângulos notáveis:} saber sen/cos de 30°, 45°, 60°
\item \textbf{Aceleração independe da massa} (sem atrito)
\item \textbf{Maior θ:} maior aceleração, menor normal
\end{enumerate}

\subsubsection{Conceitos-Chave para Memorizar}
\label{sec:org9c5c992}
\textbf{Decomposição do Peso:} - P\_x = mg sen(θ) (paralela) - P\_y = mg cos(θ) (perpendicular)

\textbf{Normal:} - N = mg cos(θ)

\textbf{Aceleração:} - Sem atrito: a = g sen(θ) - Com atrito: a = g(sen(θ) - μ cos(θ))

\subsubsection{Fórmulas Essenciais}
\label{sec:orgc627162}
\begin{verbatim}
Plano Inclinado (ângulo θ):

Decomposição do Peso:
P_x = mg sen(θ)  (paralela ao plano)
P_y = mg cos(θ)  (perpendicular)

Normal:
N = mg cos(θ)

Aceleração (sem atrito):
a = g sen(θ)

Atrito:
F_at = μN = μ mg cos(θ)

Aceleração (com atrito):
a = g(sen(θ) - μ cos(θ))

Ângulos Notáveis:
sen(30°) = 1/2 = 0,5
cos(30°) = √3/2 ≈ 0,87
sen(45°) = cos(45°) = √2/2 ≈ 0,7
sen(60°) = √3/2 ≈ 0,87
cos(60°) = 1/2 = 0,5
\end{verbatim}

\subsubsection{Resumo Visual}
\label{sec:org18e8330}
\begin{verbatim}
PLANO INCLINADO:

        N ↑
       /|
      / |
   [m]  |
    /   |
   /    ↓ P_y = mg cos(θ)
  / P_x = mg sen(θ) →
 /      
/ θ     
────────

Decomposição:
        P_y ↑
         |
         |
        θ|___→ P_x
        /
       / P ↓

P_x "desce" - sen(θ)
P_y "comprime" - cos(θ)

SEM ATRITO:       COM ATRITO:
a = g sen(θ)      a = g(sen-μcos)
\end{verbatim}

\noindent\rule{\textwidth}{0.5pt}

\textbf{Tempo de estudo recomendado:} 30 minutos \textbf{Nível de dificuldade:} Médio-Alto \textbf{Importância para a prova:} ⭐⭐⭐⭐ (muito importante - aplicação prática de forças!)

\section{12/01 - Férias Dia 6}
\label{sec:org5044d43}
\subsection{Aula 40 - Matemática: Função Logarítmica - Parte 2 (Propriedades e Equações) - 90min}
\label{sec:org8c26b12}
\subsubsection{Revisão: Logaritmos Parte 1}
\label{sec:org19ee621}
Na Aula 38, estudamos: - Definição: log\_a b = x ⟺ a\^{}x = b - Casos especiais: log\_a 1 = 0, log\_a a = 1 - Logaritmo decimal (log) e natural (ln) - Condições de existência

\textbf{Nesta aula:} Propriedades operatórias e equações logarítmicas.

\subsubsection{Propriedades Operatórias dos Logaritmos}
\label{sec:org93640de}
\paragraph{1. Logaritmo do Produto}
\label{sec:orgc3ff78e}
\begin{verbatim}
log_a (b · c) = log_a b + log_a c
\end{verbatim}

\textbf{“Logaritmo do produto = soma dos logaritmos”}

\textbf{Exemplos:} - log₂ (8 × 4) = log₂ 8 + log₂ 4 = 3 + 2 = 5 - log (100 × 10) = log 100 + log 10 = 2 + 1 = 3

\textbf{Aplicação:} log₅ 50 = log₅ (5 × 10) = log₅ 5 + log₅ 10 = 1 + log₅ 10

\paragraph{2. Logaritmo do Quociente}
\label{sec:org6994544}
\begin{verbatim}
log_a (b/c) = log_a b - log_a c
\end{verbatim}

\textbf{“Logaritmo do quociente = diferença dos logaritmos”}

\textbf{Exemplos:} - log₂ (16/4) = log₂ 16 - log₂ 4 = 4 - 2 = 2 - log (1000/10) = log 1000 - log 10 = 3 - 1 = 2

\paragraph{3. Logaritmo da Potência}
\label{sec:org7f47af6}
\begin{verbatim}
log_a (b^n) = n · log_a b
\end{verbatim}

\textbf{“Logaritmo da potência = expoente × logaritmo”}

\textbf{Exemplos:} - log₂ (8³) = 3 · log₂ 8 = 3 × 3 = 9 - log (100²) = 2 · log 100 = 2 × 2 = 4 - log₃ (√9) = log₃ (9\^{}(1/2)) = (1/2) · log₃ 9 = (1/2) × 2 = 1

\paragraph{4. Mudança de Base}
\label{sec:org13f1028}
\begin{verbatim}
log_a b = (log_c b) / (log_c a)
\end{verbatim}

\textbf{Casos particulares:}

\textbf{Para base 10:}

\begin{verbatim}
log_a b = (log b) / (log a)
\end{verbatim}

\textbf{Para base e:}

\begin{verbatim}
log_a b = (ln b) / (ln a)
\end{verbatim}

\textbf{Consequência importante:}

\begin{verbatim}
log_a b = 1 / log_b a
\end{verbatim}

\textbf{Exemplos:}

log₂ 5 = (log 5) / (log 2) ≈ 0,699 / 0,301 ≈ 2,32

log₅ 2 = 1 / log₂ 5 ≈ 1 / 2,32 ≈ 0,43

\subsubsection{Aplicações das Propriedades}
\label{sec:org6cb4b11}
\paragraph{Exemplo 1: Simplificar}
\label{sec:orgd1dd6ba}
log 2 + log 5

\textbf{Solução:} = log (2 × 5) = log 10 = 1

\paragraph{Exemplo 2: Simplificar}
\label{sec:org5710a36}
log₃ 81 - log₃ 9

\textbf{Solução:} = log₃ (81/9) = log₃ 9 = log₃ (3²) = 2

\paragraph{Exemplo 3: Calcular}
\label{sec:org72d1f02}
log₂ 5 + log₂ 8 - log₂ 10

\textbf{Solução:} = log₂ (5 × 8 / 10) = log₂ (40/10) = log₂ 4 = log₂ (2²) = 2

\paragraph{Exemplo 4: Simplificar}
\label{sec:orgb7c318f}
2 log 5 + log 4

\textbf{Solução:} = log (5²) + log 4 = log 25 + log 4 = log (25 × 4) = log 100 = 2

\subsubsection{Equações Logarítmicas}
\label{sec:orgb6a203d}
\textbf{Equação logarítmica:} incógnita dentro do logaritmo ou como logaritmo.

\paragraph{Tipo 1: log\_a f(x) = k}
\label{sec:orgc7cabd9}
\textbf{Método:} Passar para forma exponencial

\textbf{Exemplo:} log₂ (x + 1) = 3

2³ = x + 1 8 = x + 1 \textbf{x = 7}

\textbf{Verificação:} log₂ (7 + 1) = log₂ 8 = 3 ✓

\paragraph{Tipo 2: log\_a f(x) = log\_a g(x)}
\label{sec:org1bc4a4e}
\textbf{Método:} Igualar argumentos (se bases iguais)

\textbf{Exemplo:} log₅ (2x - 3) = log₅ (x + 1)

2x - 3 = x + 1 x = 4

\textbf{Verificação de condições:} 2x - 3 > 0 → x > 3/2 ✓ x + 1 > 0 → x > -1 ✓ x = 4 satisfaz ambas

\textbf{Resposta:} x = 4

\paragraph{Tipo 3: Equações com Propriedades}
\label{sec:orgdde181c}
\textbf{Exemplo 1:} log x + log (x - 3) = 1

\textbf{Solução:} log [x(x - 3)] = 1 log (x² - 3x) = 1

x² - 3x = 10¹ x² - 3x - 10 = 0 (x - 5)(x + 2) = 0

x = 5 ou x = -2

\textbf{Verificar condições:} - x > 0 ✓ (x = 5) ✗ (x = -2) - x - 3 > 0 → x > 3 ✓ (x = 5) ✗ (x = -2)

\textbf{Resposta:} x = 5

\textbf{Exemplo 2:} log₂ x - log₂ (x - 1) = 1

log₂ [x/(x - 1)] = 1

x/(x - 1) = 2¹

x = 2(x - 1) x = 2x - 2 x = 2

\textbf{Verificação:} x = 2 > 0 ✓, x - 1 = 1 > 0 ✓

\textbf{Resposta:} x = 2

\paragraph{Tipo 4: Substituição}
\label{sec:org635dd9a}
\textbf{Exemplo:} (log x)² - 3 log x + 2 = 0

\textbf{Substituição:} y = log x

y² - 3y + 2 = 0 (y - 1)(y - 2) = 0 y = 1 ou y = 2

\textbf{Voltar para x:} log x = 1 → x = 10¹ = 10 log x = 2 → x = 10² = 100

\textbf{Resposta:} x = 10 ou x = 100

\subsubsection{Inequações Logarítmicas}
\label{sec:org15a9b6d}
\textbf{Regra depende da base:}

\textbf{Se a > 1:} função crescente

\begin{verbatim}
log_a f(x) > log_a g(x)  →  f(x) > g(x)
\end{verbatim}

\textbf{Se 0 < a < 1:} função decrescente

\begin{verbatim}
log_a f(x) > log_a g(x)  →  f(x) < g(x)  (inverte!)
\end{verbatim}

\textbf{Exemplo 1:} log₂ (x + 1) > log₂ 5

Base 2 > 1 (crescente) → mantém sinal

x + 1 > 5 x > 4

\textbf{Condição:} x + 1 > 0 → x > -1

\textbf{Interseção:} x > 4

\textbf{Resposta:} S = (4, +∞)

\textbf{Exemplo 2:} log\_(1/2) x < log\_(1/2) 4

Base 1/2 < 1 (decrescente) → \textbf{inverte}

x > 4

\textbf{Condição:} x > 0

\textbf{Interseção:} x > 4

\textbf{Resposta:} S = (4, +∞)

\subsubsection{Gráfico da Função Logarítmica}
\label{sec:orga80d396}
\textbf{f(x) = log\_a x}

\textbf{Características:} - \textbf{Domínio:} x > 0 (ℝ₊*) - \textbf{Imagem:} ℝ (todos os reais) - \textbf{Intercepto com eixo x:} (1, 0) - sempre! - \textbf{Assíntota vertical:} x = 0 (eixo y) - \textbf{a > 1:} crescente - \textbf{0 < a < 1:} decrescente

\textbf{Relação com exponencial:}

\begin{verbatim}
Se f(x) = a^x
Então f⁻¹(x) = log_a x
\end{verbatim}

Gráficos são simétricos em relação à reta y = x.

\subsubsection{Exercícios Resolvidos}
\label{sec:orgb88b6e0}
\paragraph{Exercício 1}
\label{sec:org322ac43}
Simplifique: log₅ 25 + log₅ 5 - log₅ 125

\textbf{Solução:} = 2 + 1 - 3 = 0

\paragraph{Exercício 2}
\label{sec:orgecc708d}
Calcule: log 2 + log 50

\textbf{Solução:} = log (2 × 50) = log 100 = 2

\paragraph{Exercício 3}
\label{sec:org78fce06}
Resolva: log₃ (x² - 8) = 2

\textbf{Solução:} x² - 8 = 3² x² - 8 = 9 x² = 17 x = ±√17

\textbf{Condição:} x² - 8 > 0 → x² > 8

Ambos √17 e -√17 satisfazem (pois 17 > 8)

\textbf{Resposta:} x = √17 ou x = -√17

\paragraph{Exercício 4}
\label{sec:org8dd0315}
Resolva: log x + log (x + 3) = 1

\textbf{Solução:} log [x(x + 3)] = 1 x² + 3x = 10 x² + 3x - 10 = 0 (x + 5)(x - 2) = 0

x = -5 ou x = 2

\textbf{Condições:} x > 0: apenas x = 2 ✓ x + 3 > 0: x > -3, ambos satisfazem, mas x = -5 já foi eliminado

\textbf{Resposta:} x = 2

\paragraph{Exercício 5}
\label{sec:orgd90a07e}
Calcule log₈ 2.

\textbf{Solução:}

\textbf{Método 1 - Definição:} 8\^{}x = 2 (2³)\^{}x = 2¹ 2\^{}(3x) = 2¹ 3x = 1 x = 1/3

\textbf{Método 2 - Mudança de base:} log₈ 2 = (log 2)/(log 8) = (log 2)/(log 2³) = (log 2)/(3 log 2) = 1/3

\textbf{Resposta:} 1/3

\subsubsection{Dicas para a Prova}
\label{sec:orgc361c35}
\begin{enumerate}
\item \textbf{Produto:} log(ab) = log a + log b
\item \textbf{Quociente:} log(a/b) = log a - log b
\item \textbf{Potência:} log(a\^{}n) = n log a
\item \textbf{Mudança de base:} log\_a b = (log b)/(log a)
\item \textbf{Equação log\_a f = log\_a g:} f = g (igualar)
\item \textbf{Inequação e a > 1:} mantém sinal
\item \textbf{Inequação e 0 < a < 1:} inverte sinal
\item \textbf{Sempre verificar condições:} logaritmando > 0
\end{enumerate}

\subsubsection{Conceitos-Chave para Memorizar}
\label{sec:org3545930}
\textbf{Propriedades:} - Produto: log(ab) = log a + log b - Quociente: log(a/b) = log a - log b - Potência: log(a\^{}n) = n log a - Mudança: log\_a b = (log b)/(log a)

\textbf{Equações:} - log\_a f(x) = k → a\^{}k = f(x) - log\_a f = log\_a g → f = g

\textbf{Inequações:} - a > 1: crescente (mantém) - 0 < a < 1: decrescente (inverte)

\textbf{Sempre:} verificar condições (> 0)

\subsubsection{Fórmulas Essenciais}
\label{sec:org115e960}
\begin{verbatim}
Propriedades Operatórias:
log_a (bc) = log_a b + log_a c
log_a (b/c) = log_a b - log_a c
log_a (b^n) = n · log_a b

Mudança de Base:
log_a b = (log_c b) / (log_c a)
log_a b = (log b) / (log a)  [base 10]
log_a b = (ln b) / (ln a)    [base e]
log_a b = 1 / log_b a

Equação:
log_a f(x) = log_a g(x)  ⟹  f(x) = g(x)

Inequação:
a > 1:     log_a f > log_a g  ⟹  f > g
0 < a < 1: log_a f > log_a g  ⟹  f < g

Condições:
Sempre verificar logaritmando > 0
\end{verbatim}

\subsubsection{Resumo Visual}
\label{sec:org69def76}
\begin{verbatim}
PROPRIEDADES:

Produto:      Quociente:
log(a×b)      log(a/b)
  =             =
log a + log b log a - log b

Potência:
log(a^n) = n·log a

EQUAÇÃO:
log_a f = log_a g
     ↓
    f = g

INEQUAÇÃO:
a > 1 (cresce):   0 < a < 1 (decresce):
log_a f > log_a g log_a f > log_a g
     ↓                 ↓
    f > g             f < g
  (mantém)          (inverte)
\end{verbatim}

\noindent\rule{\textwidth}{0.5pt}

\textbf{Tempo de estudo recomendado:} 90 minutos \textbf{Nível de dificuldade:} Médio-Alto \textbf{Importância para a prova:} ⭐⭐⭐⭐⭐ (essencial - propriedades são muito cobradas!)

\subsection{Aula 41 - Química: Propriedades das Substâncias Segundo Ligações - 30min}
\label{sec:orgd6ce916}
\subsubsection{Revisão: Ligações Químicas}
\label{sec:orgb659d3f}
Já estudamos os três tipos principais: - \textbf{Aula 31:} Ligação Iônica (metal + não-metal, íons) - \textbf{Aula 33:} Ligação Covalente (não-metal + não-metal, moléculas) - \textbf{Aula 35:} Ligação Metálica (metais, mar de elétrons)

\textbf{Nesta aula:} Como o tipo de ligação determina as propriedades das substâncias.

\subsubsection{Propriedades dos Compostos Iônicos}
\label{sec:org69fadce}
\textbf{Exemplos:} NaCl, CaO, MgCl₂, K₂SO₄

\paragraph{Características Físicas}
\label{sec:org2d0de52}
\textbf{1. Estado Físico:} - \textbf{Sólidos} à temperatura ambiente - Formam cristais (retículo cristalino ordenado)

\textbf{2. Pontos de Fusão e Ebulição:} - \textbf{Muito altos} - Fortes atrações eletrostáticas entre íons

\textbf{Exemplos:} - NaCl: PF = 801°C - MgO: PF = 2852°C

\textbf{3. Dureza:} - \textbf{Duros} (resistem a deformação) - \textbf{Quebradiços} (fraturam ao invés de deformar)

\paragraph{Condutividade Elétrica}
\label{sec:org78c2606}
\textbf{Sólidos:} NÃO conduzem - Íons fixos no retículo, não se movem

\textbf{Fundidos (líquidos):} CONDUZEM - Íons livres para se mover

\textbf{Dissolvidos em água:} CONDUZEM - Íons dispersos na solução, móveis

\textbf{Resumo:}

\begin{verbatim}
Iônico sólido: NÃO conduz
Iônico fundido/dissolvido: CONDUZ
\end{verbatim}

\paragraph{Solubilidade}
\label{sec:org809fad5}
\textbf{Em água (polar):} Geralmente \textbf{solúveis} - “Semelhante dissolve semelhante” - Compostos iônicos são polares - Íons se dispersam na água

\textbf{Em solventes apolares:} \textbf{Insolúveis} - Gasolina, benzeno, etc.

\textbf{Regras de solubilidade (principais):} - Sais de Na⁺, K⁺, NH₄⁺: sempre solúveis - Nitratos (NO₃⁻): sempre solúveis - Cloretos, brometos, iodetos: geralmente solúveis (exceto Ag⁺, Pb²⁺) - Sulfatos: geralmente solúveis (exceto Ba²⁺, Pb²⁺)

\subsubsection{Propriedades dos Compostos Covalentes (Moleculares)}
\label{sec:orgd3c517e}
\textbf{Exemplos:} H₂O, CO₂, CH₄, O₂, N₂

\paragraph{Características Físicas}
\label{sec:org579f68a}
\textbf{1. Estado Físico:} - Maioria: \textbf{gases} ou \textbf{líquidos} à temp. ambiente - Alguns sólidos (gelo, açúcar, naftaleno)

\textbf{2. Pontos de Fusão e Ebulição:} - \textbf{Baixos} (comparados a iônicos) - Forças intermoleculares fracas

\textbf{Exemplos:} - H₂O: PE = 100°C - CO₂: sublima a -78°C

\textbf{Exceção:} Substâncias covalentes em rede (SiO₂, diamante) - PF/PE muito altos - Ligações covalentes formam rede tridimensional

\paragraph{Condutividade Elétrica}
\label{sec:orgfb3d3d5}
\textbf{Geralmente:} NÃO conduzem - Não têm íons ou elétrons livres

\textbf{Exceções:} - \textbf{Grafite:} estrutura especial com elétrons livres - \textbf{Ácidos em água:} formam íons (HCl → H⁺ + Cl⁻)

\paragraph{Solubilidade}
\label{sec:org7009d72}
Depende da \textbf{polaridade} da molécula:

\textbf{Moléculas polares:} - Solúveis em solventes polares (água) - Exemplos: H₂O, HCl, NH₃, álcoois

\textbf{Moléculas apolares:} - Solúveis em solventes apolares (gasolina, benzeno) - Exemplos: O₂, N₂, CH₄, CO₂, gorduras

\textbf{Regra de ouro:}

\begin{verbatim}
"Semelhante dissolve semelhante"
\end{verbatim}

\subsubsection{Propriedades das Substâncias Metálicas}
\label{sec:orgb17c66f}
\textbf{Exemplos:} Fe, Cu, Au, Al, Na

\paragraph{Características}
\label{sec:org377dcc5}
\textbf{1. Estado Físico:} - \textbf{Sólidos} à temp. ambiente (exceto Hg - mercúrio)

\textbf{2. Pontos de Fusão:} - \textbf{Variados} (baixos a muito altos) - W (tungstênio): 3422°C (alto) - Hg (mercúrio): -39°C (líquido) - Ga (gálio): 30°C (derrete na mão)

\textbf{3. Condutividade Elétrica e Térmica:} - \textbf{Excelentes condutores} (sempre) - Elétrons livres transportam carga e calor

\textbf{Melhores condutores:} - Ag (prata) > Cu (cobre) > Au (ouro) > Al (alumínio)

\textbf{4. Brilho Metálico:} - Superfície reflete luz (elétrons livres)

\textbf{5. Maleabilidade e Ductilidade:} - \textbf{Maleáveis:} podem ser martelados (lâminas) - \textbf{Dúcteis:} podem ser esticados (fios)

\textbf{6. Densidade:} - Maioria são densos - Exceções: Li, Na, K (menos densos que água)

\subsubsection{Comparação das Propriedades}
\label{sec:org151622f}
\begin{verbatim}
┌──────────────┬─────────┬──────────┬──────────┐
│  Propriedade │ IÔNICA  │COVALENTE │ METÁLICA │
├──────────────┼─────────┼──────────┼──────────┤
│Estado físico │ Sólidos │Gases/Líq.│ Sólidos  │
├──────────────┼─────────┼──────────┼──────────┤
│   PF / PE    │  Altos  │  Baixos  │ Variados │
├──────────────┼─────────┼──────────┼──────────┤
│Condutividade │Fund/Dis.│   Não    │   Sim    │
│  Elétrica    │         │          │ (sempre) │
├──────────────┼─────────┼──────────┼──────────┤
│ Solubilidade │  H₂O    │  Varia   │  Baixa   │
│              │(polar)  │(polar/ap)│          │
├──────────────┼─────────┼──────────┼──────────┤
│    Dureza    │Duros,   │  Varia   │  Varia   │
│              │quebradiç.│          │(maleáv.) │
├──────────────┼─────────┼──────────┼──────────┤
│   Exemplo    │  NaCl   │   H₂O    │    Fe    │
└──────────────┴─────────┴──────────┴──────────┘
\end{verbatim}

\subsubsection{Aplicações Práticas}
\label{sec:org24c6706}
\paragraph{1. Sal de Cozinha (NaCl) - Iônico}
\label{sec:org208ebea}
\begin{itemize}
\item Sólido cristalino
\item Alto PF (801°C)
\item Solúvel em água
\item Solução conduz eletricidade (íons Na⁺ e Cl⁻)
\end{itemize}

\paragraph{2. Água (H₂O) - Covalente Polar}
\label{sec:orgebbf7dd}
\begin{itemize}
\item Líquida à temp. ambiente
\item PE = 100°C (baixo)
\item Polar: dissolve iônicos e polares
\item Não conduz pura (mas conduz com íons dissolvidos)
\end{itemize}

\paragraph{3. Cobre (Cu) - Metálico}
\label{sec:org0ddf5bf}
\begin{itemize}
\item Sólido à temp. ambiente
\item Excelente condutor elétrico (fios)
\item Maleável e dúctil
\item Brilho metálico
\end{itemize}

\subsubsection{Exercícios Resolvidos}
\label{sec:orgb6dbdbb}
\paragraph{Exercício 1}
\label{sec:org6014bb8}
Por que o sal de cozinha (NaCl) não conduz eletricidade quando sólido, mas conduz quando dissolvido em água?

\textbf{Resposta:} \textbf{Sólido:} íons Na⁺ e Cl⁻ estão fixos no retículo cristalino, não podem se mover, logo NÃO conduzem.

\textbf{Dissolvido:} íons ficam livres na solução aquosa, podem se mover e transportar carga elétrica, logo CONDUZEM.

\paragraph{Exercício 2}
\label{sec:org83e6472}
Explique por que metais são bons condutores de eletricidade.

\textbf{Resposta:} Metais têm \textbf{elétrons livres} (mar de elétrons) que se movem facilmente. Ao aplicar diferença de potencial, os elétrons fluem, criando corrente elétrica.

\paragraph{Exercício 3}
\label{sec:org7c17c10}
Uma substância é sólida à temperatura ambiente, tem alto ponto de fusão, é solúvel em água e sua solução conduz eletricidade. Que tipo de ligação tem?

\textbf{Resposta:} \textbf{Ligação iônica}. As características (sólido, alto PF, solúvel em água, solução condutora) são típicas de compostos iônicos.

\paragraph{Exercício 4}
\label{sec:orgf8bb268}
(UFMG) Compare NaCl e H₂O quanto à condutividade elétrica.

\textbf{Resposta:} - \textbf{NaCl sólido:} não conduz (íons fixos) - \textbf{NaCl fundido/dissolvido:} conduz (íons livres) - \textbf{H₂O pura:} não conduz (covalente, sem íons) - \textbf{H₂O com sais dissolvidos:} conduz (íons do sal)

\subsubsection{Dicas para a Prova}
\label{sec:org5d55f79}
\begin{enumerate}
\item \textbf{Iônico:} sólidos, altos PF/PE, conduzem fundidos/dissolvidos
\item \textbf{Covalente:} gases/líquidos, baixos PF/PE, não conduzem
\item \textbf{Metálico:} sólidos, conduzem sempre, maleáveis
\item \textbf{Solubilidade:} “semelhante dissolve semelhante”
\item \textbf{Condutividade:} precisa de partículas carregadas móveis
\item \textbf{Iônicos:} solúveis em água (polar)
\item \textbf{Covalentes apolares:} solúveis em apolares
\item \textbf{Metais:} sempre conduzem (elétrons livres)
\end{enumerate}

\subsubsection{Conceitos-Chave para Memorizar}
\label{sec:org347ab19}
\textbf{Iônica:} - Sólidos cristalinos - Altos PF/PE - Conduzem fundidos/dissolvidos - Solúveis em água

\textbf{Covalente:} - Gases/líquidos (maioria) - Baixos PF/PE - Não conduzem (geralmente) - Solubilidade varia

\textbf{Metálica:} - Sólidos (exceto Hg) - Conduzem sempre - Maleáveis, dúcteis - Brilho metálico

\subsubsection{Resumo Visual}
\label{sec:org4d70b66}
\begin{verbatim}
PROPRIEDADES × LIGAÇÕES:

IÔNICA (NaCl):
├─ Sólido cristalino
├─ Alto PF (801°C)
├─ Conduz fundido/dissolvido
└─ Solúvel H₂O

COVALENTE (H₂O):
├─ Líquido/Gás
├─ Baixo PE (100°C)
├─ Não conduz puro
└─ Polar: dissolve polar

METÁLICA (Cu):
├─ Sólido
├─ Conduz sempre
├─ Maleável, dúctil
└─ Brilho

TABELA:
           Iônico  Cov.  Met.
Estado     Sól.    G/L   Sól.
PF/PE      Alto    Baixo Var.
Conduz     F/D     Não   Sim
\end{verbatim}

\noindent\rule{\textwidth}{0.5pt}

\textbf{Tempo de estudo recomendado:} 30 minutos \textbf{Nível de dificuldade:} Médio \textbf{Importância para a prova:} ⭐⭐⭐⭐ (importante - relaciona estrutura e propriedades!)

\section{12/02 - Férias Dia 7 (ÚLTIMO DIA DE FÉRIAS!)}
\label{sec:orgbc26264}
\subsection{Aula 42 - Matemática: Relação entre Funções Exponencial e Logarítmica - 60min}
\label{sec:orgf59d310}
\subsubsection{Funções Inversas}
\label{sec:orgdd0af8d}
\textbf{Definição:} Duas funções f e g são inversas se:

\begin{verbatim}
f(g(x)) = x  e  g(f(x)) = x
\end{verbatim}

\textbf{Notação:} g = f⁻¹ (função inversa de f)

\textbf{Graficamente:} Gráficos são simétricos em relação à reta y = x

\subsubsection{Função Exponencial e Logarítmica: Inversas}
\label{sec:org47158a4}
\textbf{Função exponencial:} f(x) = a\^{}x \textbf{Função logarítmica:} g(x) = log\_a x

\textbf{São funções inversas!}

\paragraph{Verificação}
\label{sec:orgac7364f}
\textbf{1. f(g(x)) = x:}

\begin{verbatim}
f(g(x)) = f(log_a x)
        = a^(log_a x)
        = x  ✓
\end{verbatim}

\textbf{2. g(f(x)) = x:}

\begin{verbatim}
g(f(x)) = g(a^x)
        = log_a (a^x)
        = x  ✓
\end{verbatim}

\subsubsection{Propriedades Fundamentais}
\label{sec:org1dc4880}
\textbf{Decorrentes da relação inversa:}

\textbf{1. Composição:}

\begin{verbatim}
a^(log_a x) = x  (x > 0)
log_a (a^x) = x  (x ∈ ℝ)
\end{verbatim}

\textbf{2. Domínio e Imagem:}

\begin{center}
\begin{tabular}{lll}
Função & Domínio & Imagem\\[0pt]
\hline
f(x) = a\^{}x & ℝ & ℝ₊* (0, +∞)\\[0pt]
g(x) = log\_a x & ℝ₊* (0, +∞) & ℝ\\[0pt]
\end{tabular}
\end{center}

\textbf{Observe:} Domínio de uma é imagem da outra!

\textbf{3. Interceptos:}

\textbf{Exponencial:} - Intercepto y: (0, 1) - sempre! - Não intercepta eixo x (assíntota)

\textbf{Logarítmica:} - Intercepto x: (1, 0) - sempre! - Não intercepta eixo y (assíntota)

\subsubsection{Gráficos: Simetria}
\label{sec:org5bb0585}
\textbf{Propriedade:} Gráficos de f(x) = a\^{}x e g(x) = log\_a x são \textbf{simétricos} em relação à \textbf{reta y = x}.

\paragraph{Exemplo: a = 2}
\label{sec:org95d15b6}
\textbf{f(x) = 2\^{}x:} - Passa por (0, 1), (1, 2), (2, 4) - Crescente - Assíntota: y = 0 (eixo x)

\textbf{g(x) = log₂ x:} - Passa por (1, 0), (2, 1), (4, 2) - Crescente - Assíntota: x = 0 (eixo y)

\textbf{Gráfico:}

\begin{verbatim}
  |    y=x (diagonal)
4 |   /    •2^x
2 | •/   •
1 |•/──•─────
  |/ •log₂x
  |•____________
    1  2  4
\end{verbatim}

\textbf{Simetria:} Pontos (a, b) em f correspondem a (b, a) em g.

\subsubsection{Equações Envolvendo Ambas}
\label{sec:org1407021}
\paragraph{Tipo 1: a\^{}x = k → Usar log}
\label{sec:org7ed25ce}
\textbf{Exemplo:} 2\^{}x = 5

\textbf{Método 1 - Logaritmo decimal:}

\begin{verbatim}
log(2^x) = log 5
x log 2 = log 5
x = (log 5)/(log 2)
x ≈ 0,699/0,301 ≈ 2,32
\end{verbatim}

\textbf{Método 2 - Logaritmo na base adequada:}

\begin{verbatim}
log₂(2^x) = log₂ 5
x = log₂ 5
\end{verbatim}

\paragraph{Tipo 2: log\_a x = k → Usar exponencial}
\label{sec:org2b3a7a0}
\textbf{Exemplo:} log₃ x = 4

\begin{verbatim}
3⁴ = x
x = 81
\end{verbatim}

\paragraph{Tipo 3: Mistas}
\label{sec:orgd0e6492}
\textbf{Exemplo:} 2\^{}x = log₂ 16

\textbf{Resolver log₂ 16 primeiro:} log₂ 16 = log₂ (2⁴) = 4

\textbf{Então:} 2\^{}x = 4 = 2² x = 2

\subsubsection{Mudança de Base: Aplicações}
\label{sec:org25d95af}
\textbf{Fórmula:}

\begin{verbatim}
log_a b = (log_c b)/(log_c a)
\end{verbatim}

\textbf{Uso prático:} Calcular logaritmos em bases diferentes de 10 ou e.

\textbf{Exemplo:} Calcular log₅ 20

\begin{verbatim}
log₅ 20 = (log 20)/(log 5)
        ≈ 1,301/0,699
        ≈ 1,86
\end{verbatim}

\textbf{Ou usando ln:}

\begin{verbatim}
log₅ 20 = (ln 20)/(ln 5)
        ≈ 2,996/1,609
        ≈ 1,86
\end{verbatim}

\subsubsection{Aplicações Práticas}
\label{sec:org28422f5}
\paragraph{Problema 1: Crescimento Exponencial}
\label{sec:org91db6bc}
Uma população de 1000 bactérias dobra a cada hora. Quando atinge 10.000?

\textbf{Modelo:} P(t) = 1000 · 2\^{}t

\textbf{Resolver:} 1000 · 2\^{}t = 10000 2\^{}t = 10 t = log₂ 10 t = (log 10)/(log 2) = 1/0,301 ≈ 3,32 horas

\textbf{Resposta:} Aproximadamente 3,3 horas

\paragraph{Problema 2: Decaimento Radioativo}
\label{sec:org9e99145}
Substância com meia-vida de 5 anos. Tempo para restar 10\% da massa inicial?

\textbf{Modelo:} M(t) = M₀ · (1/2)\^{}(t/5)

\textbf{Resolver:} (1/2)\^{}(t/5) = 0,1

\textbf{Aplicar log:} log[(1/2)\^{}(t/5)] = log 0,1 (t/5) log(1/2) = log 0,1 (t/5) × (-0,301) = -1 t/5 = 1/0,301 ≈ 3,32 t ≈ 16,6 anos

\paragraph{Problema 3: Juros Compostos}
\label{sec:orgcf0f283}
Capital de R\$ 2000 a 8\% ao ano. Quando duplica?

\textbf{Modelo:} M = 2000(1,08)\^{}t

\textbf{Resolver:} 2000(1,08)\^{}t = 4000 1,08\^{}t = 2 t = log₁.₀₈ 2 t = (log 2)/(log 1,08) t ≈ 0,301/0,0334 ≈ 9 anos

\subsubsection{Exercícios Resolvidos}
\label{sec:orgf3b2981}
\paragraph{Exercício 1}
\label{sec:orgd2f1c9d}
Resolva: 3\^{}x = 20

\textbf{Solução:} x = log₃ 20 = (log 20)/(log 3) ≈ 1,301/0,477 ≈ 2,73

\textbf{Resposta:} x ≈ 2,73

\paragraph{Exercício 2}
\label{sec:orgdabda23}
Se log₂ x = 5, qual o valor de x?

\textbf{Solução:} 2⁵ = x x = 32

\textbf{Resposta:} 32

\paragraph{Exercício 3}
\label{sec:orgd7c7c65}
Simplifique: 5\^{}(log₅ 7)

\textbf{Solução:} Propriedade: a\^{}(log\_a b) = b

5\^{}(log₅ 7) = 7

\textbf{Resposta:} 7

\paragraph{Exercício 4}
\label{sec:orgb3227dd}
Calcule: log₂ (2\^{}(3x)) quando x = 2

\textbf{Solução:} log₂ (2\^{}(3x)) = 3x = 3 × 2 = 6

\textbf{Resposta:} 6

\subsubsection{Dicas para a Prova}
\label{sec:orgdc88027}
\begin{enumerate}
\item \textbf{a\^{}(log\_a b) = b} (propriedade inversa)
\item \textbf{log\_a (a\^{}x) = x} (propriedade inversa)
\item \textbf{Simetria:} gráficos refletem em y = x
\item \textbf{Equação a\^{}x = k:} aplicar log
\item \textbf{Equação log x = k:} passar para exponencial
\item \textbf{Mudança de base:} log\_a b = (log b)/(log a)
\item \textbf{Domínio exp:} ℝ; \textbf{Imagem exp:} (0, +∞)
\item \textbf{Domínio log:} (0, +∞); \textbf{Imagem log:} ℝ
\end{enumerate}

\subsubsection{Conceitos-Chave para Memorizar}
\label{sec:org34957e8}
\textbf{Funções Inversas:} - f(x) = a\^{}x ⟺ f⁻¹(x) = log\_a x - Composição = identidade - Gráficos simétricos (y = x)

\textbf{Propriedades:} - a\^{}(log\_a x) = x - log\_a (a\^{}x) = x

\textbf{Domínio/Imagem:} - Exp: D = ℝ, Im = (0,+∞) - Log: D = (0,+∞), Im = ℝ

\textbf{Pontos Especiais:} - Exp: (0, 1) - Log: (1, 0)

\subsubsection{Fórmulas Essenciais}
\label{sec:orgbf35ded}
\begin{verbatim}
Relação Inversa:
f(x) = a^x  ⟺  f⁻¹(x) = log_a x

Composição:
a^(log_a x) = x  (x > 0)
log_a (a^x) = x  (x ∈ ℝ)

Resolver a^x = k:
x = log_a k = (log k)/(log a)

Resolver log_a x = k:
x = a^k

Mudança de Base:
log_a b = (log b)/(log a)
\end{verbatim}

\subsubsection{Resumo Visual}
\label{sec:org9f31ce1}
\begin{verbatim}
SIMETRIA:

    f(x)=a^x    y=x    g(x)=log_a x
         \       /       /
          \     /       /
           \   /       /
         (0,1)/__(1,0)
             /
            /
           /

Pontos correspondentes:
Exp: (0,1), (1,a), (2,a²)
Log: (1,0), (a,1), (a²,2)

RELAÇÃO:
         Exponencial
              ↓
         y = a^x
              ↓
         Aplica log_a
              ↓
         x = log_a y
              ↓
         Logarítmica

PROPRIEDADES INVERSAS:
a^(log_a x) = x
log_a (a^x) = x
\end{verbatim}

\noindent\rule{\textwidth}{0.5pt}

\textbf{Tempo de estudo recomendado:} 60 minutos \textbf{Nível de dificuldade:} Médio-Alto \textbf{Importância para a prova:} ⭐⭐⭐⭐⭐ (essencial - relação fundamental!)

\noindent\rule{\textwidth}{0.5pt}

\section{🎉 PERÍODO DE FÉRIAS COMPLETO! 🎉}
\label{sec:org902b8cf}
Você concluiu todas as \textbf{14 aulas} do período de férias (26/11 a 02/12)!

\textbf{Progresso total:} 43/96 lições concluídas (44,8\%)

\textbf{O que você estudou nas férias:} - ✅ Matemática: Função Quadrática (aprofundamento), Função Exponencial (completa), Função Logarítmica (completa), Relação Exp-Log - ✅ Química: Ligações Químicas completas (Iônica, Covalente, Metálica) + Propriedades - ✅ Física: Forças especiais e Plano Inclinado

\textbf{Próximos passos:} - Semana 2: 03/12 a 07/12 (Geometria, Hidrostática, Estequiometria, Geografia, Humanas, Biologia) - Semana 3 (final): 09/12 a 13/12 (Revisões intensivas) - \textbf{PROVA: 14/12 - SÁBADO}

\textbf{Você está quase na metade! Continue com dedicação! 💪📚}
\end{document}
